\documentclass[main.tex,fontsize=8pt,paper=a4,paper=landscape,DIV=calc,]{scrartcl}
\input{../ost-summary-template.tex}

\renewcommand{\sectioncolor}[1]{\colorbox{black!60}{\parbox{0.97\linewidth}{\color{white}#1}}}
\renewcommand{\subsectioncolor}[1]{\colorbox{black!50}{\parbox{0.97\linewidth}{\color{white}#1}}}
\renewcommand{\subsubsectioncolor}[1]{\colorbox{black!40}{\parbox{0.97\linewidth}{\color{white}#1}}}
\renewcommand{\paragraphcolor}[1]{\colorbox{black!30}{\parbox{0.97\linewidth}{\color{white}#1}}}
\renewcommand{\subparagraphcolor}[1]{\colorbox{black!20}{\parbox{0.97\linewidth}{\color{white}#1}}}

\begin{document}
\begin{multicols*}{4}

\lstset{
    language={[x86masm]Assembler},
    style=code,
}

\section{Assembly}
Assembly is a platform -> intel, arm, risc-v dependent programming language. 
It converts instructions into binary -> machine code. Most compiled language will convert to assembly code before being compiled to binary!

\subsection{Numbers in Assembly}
\begin{itemize}
\item \textcolor{black}{db 48 -> Byte 48d}
\item \textcolor{black}{db 0x35, 0h21, 049h -> Bytes 35h, 21h, 49h}
\item \textcolor{black}{db 'a' -> ASCII-Code a == 61h == db 0x61}
\item \textcolor{purple}{db == Byte -> 8 Bit}
\item \textcolor{purple}{dw == Word -> 16 Bit == db first-nr, db second-nr}
\item \textcolor{purple}{dd == Doubleword -> 32 Bit}
\item \textcolor{purple}{dq == QuadWord -> 64 Bit}
\item \textcolor{orange}{048d -> d for decimal}
\item \textcolor{orange}{048h or 0x48 -> h for hex}
\item \textcolor{orange}{10001000b or 1000\_1000b -> b/y for binary}
\end{itemize}
\subsection{Length Calculation}
\includegraphics[scale=0.2]{2022-09-27-04:27:52.png}
first define a quadword with 2 unknown labels, then define my\_text with 'Besys',
after define aftermytext and recude the current instruction pointer with the instruction pointer at my\_text
\subsection{Default Registers}
\begin{itemize}
\item Right 8 bit register: \textcolor{purple}{AL}
\item Left 8 bit register: \textcolor{purple}{AH}
\item 16 bit register: \textcolor{purple}{AX}
\item 32 bit register: \textcolor{purple}{EAX}
\item 64 bit register: \textcolor{purple}{RAX}
\end{itemize} 

\subsection{Special Registers}
\begin{itemize}
  \item \textcolor{purple}{RAX} general use register
  \item \textcolor{purple}{RCX} Counter for loops or strings
  \item \textcolor{purple}{RDX} Pointer for I/O Operations
  \item \textcolor{purple}{RBX} Datapointer
  \item \textcolor{purple}{RSI,RDI} Source for stringoperations
  \item \textcolor{purple}{RSP} Stackpointer!!
  \item \textcolor{purple}{RBP} Basepointer
  \item \textcolor{purple}{R8-R15} additional registers
\end{itemize}
Note other registers like RBX exist as well, but aren't specifically used for something.

\subsection{Dealing with Memoryi}
Memory is accessed with the [] operators, this can also be offset with either multiplication or other calulations
\begin{lstlisting}
mov rax, [0x8000] ; ok move 8000h into rax
mov [0x8000], rax ; ok move value at rax into 0x8000
mov rax, rbx      ; ok move rbx into rax, no memory access!!
mov [0x8000], [0x7000] ; error can't move memory to memory!!
\end{lstlisting}
\vspace{2mm}

\subsection{Assembly Instructions}
\begin{itemize}
\item mov rax, 1 \textcolor{teal}{//move the value 1 into rax, keep in mind that mov can hold other operations!}
\item equ rax, 1+1 \textcolor{teal}{//arithmic operation}
\item add z,   q  \textcolor{teal}{// z + q} 
\item sub z,   q  \textcolor{teal}{// z - q}
\item adc z,   q  \textcolor{teal}{// z + q + c (carry -> previous calculation)}
\item sbb z,   q  \textcolor{teal}{// z - q - c (carry -> previous calculation)}
\item neg z       \textcolor{teal}{// 0 - z ("zweierkomplement")}
\item inc z       \textcolor{teal}{// z++ }
\item dec z       \textcolor{teal}{// z-- }
\item mul z,      \textcolor{teal}{// multiply with implicit 2.operand }\newline
mul rbx -> RDX:RAX <-- RAX * RBX
\item imul z,   i  \textcolor{teal}{// signed equivalent for mul, z * i }
\item div z,      \textcolor{teal}{// divide with implicit 2.operand}\newline
div rbx  \newline
d = RDX:RAX\newline
RAX <-- RDX:RAX / RBX\newline
RDX <-- RDX:RAX mod RBX
\item shl z,   i  \textcolor{teal}{// z * \(2^i\)               --> shift}
\item shr z,   i  \textcolor{teal}{// z * \(2^{-i}\) z signed   --> shift}
\item sar z,   i  \textcolor{teal}{// z * \(2^{-i}\) z unsigned --> shift}
\item rol z,   i  \textcolor{teal}{// Left-Rotate i Bits }
\item ror z,   i  \textcolor{teal}{// Right-Rotate i Bits }
\item and rax, rbx\textcolor{teal}{// AND}
\item not rax     \textcolor{teal}{// NOT}
\item cmp rax, 3 \textcolor{teal}{// compare rax and 3, set flag if not}
\item cmov rax, 5 \textcolor{teal}{// move 5 into rax if condition met}
\item je 230 \textcolor{teal}{// move 230 down if condition met}
\end{itemize}

\subsubsection{Flags}
\textcolor{purple}{Carry Flag (CF) == overflow with unsigned intergers}\newline
0000 + 1111 = 0000, CF = 1 --> 1 + 15 = 0, CF = 1
\textcolor{purple}{OverFlow Flag (OF) == overflow with signed integers}\newline
0011 + 0001 = 1000 -> 7 + 1 = -8 (negative prefix)
\textcolor{purple}{Zero Flag (ZF) == set when result is 0}
\textcolor{purple}{Sign Flag == is the highest bit of the result}
\textcolor{purple}{Parity Flag (PF) == set if lowest bute has an even number of bits}

\subsubsection{Usage of compare with Condition Codes}
\textcolor{teal}{A : Above } \textcolor{teal}{ -> CF = 0 AND ZF = 0}\newline
\textcolor{teal}{AE: Above or Equal } \textcolor{teal}{ -> CF = 0}\newline
\textcolor{teal}{B : Below } \textcolor{teal}{ -> CF = 1}\newline
\textcolor{teal}{BE: Below or Equal } \textcolor{teal}{ -> CF = 1 AND ZF = 1}\newline
\textcolor{teal}{E : Equal } \textcolor{teal}{ -> ZF = 1}\newline
\textcolor{teal}{G : Greater } \textcolor{teal}{ -> SF = OF = 0 AND ZF = 0}\newline
\textcolor{teal}{GE: Greater or Equal } \textcolor{teal}{ -> SF = OF}\newline
\textcolor{teal}{L : Less } \textcolor{teal}{ -> SF != OF}\newline
\textcolor{teal}{LE: Less or Equal } \textcolor{teal}{ -> SF != OF AND ZF = 1}\newline
\textcolor{teal}{PE: Parity Even}  \textcolor{teal}{ -> PF =1}\newline
\textcolor{teal}{PO: Parity Old } \textcolor{teal}{ -> PF = 0}\newline
\textcolor{teal}{Z:  Zero} \textcolor{teal}{ -> ZF = 1}

\subsection{Syscall}
Syscalls are special operations that the Operating system understands
\begin{lstlisting}
mov rax, 60 // 60 == exit instruction
mov rdi, 0 // exit code -> 0 means ok
syscall // OS executes instruction from rax
\end{lstlisting}
\vspace{2mm}



\lstset{
    language=c,
    style=code,
}

\section{C}


\section{Cache}


\section{RAM}

\end{multicols*}
\end{document}
