\documentclass[main.tex,fontsize=12pt,paper=a4,paper=landscape,DIV=calc,]{scrartcl}
\input{../template-exam.tex}


\lstset{
    language=[Sharp]C,
    style=code,
}

\begin{document}
\begin{multicols*}{4}

\section{vector geometry}
Point: P (5,2)\newline
vector: \(\vec{v} \left(\dfrac{5}{2}\right)\) \footnotesize not used here to save space
\normalsize

\subsection{Translation | linear}
P + \(\vec{v} \) = (1,2,3) + (4,5,6) = (5,6,9) 

\subsection{Scaling | linear}
P * s = (s * 1, s * 2, s * 3)

\subsection{Rotation | NOT linear}
\(R_\theta \left(\dfrac{x}{0}\right) = \left(\dfrac{cos \theta}{sin \theta} x\right)\)\newline
\(R_\theta \left(\dfrac{0}{y}\right) = \left(\dfrac{-sin \theta}{cos \theta} y\right)\)\newline
\(R_\theta \left(\dfrac{x}{y}\right) = \left(\dfrac{x * cos \theta - y * sin \theta}{x * sin \theta + y * cos \theta} x\right)\)

\subsection{Multiplication of Vectors}
\textbf{Multiplication of vectors is not communitative!}\newline
\(\left[\begin{smallmatrix}\textcolor{blue}{x_{11} \, x_{12} \, x_{13}} \\ x_{21} \, x_{22} \, x_{23} \end{smallmatrix}\right] *
\left[\begin{smallmatrix} \textcolor{green}{y_{11}} \, y_{12} \\ \textcolor{green}{y_{21}} \, y_{22} \\ 
\textcolor{green}{y_{31}} \, y_{32}\end{smallmatrix}\right] =
\left[\begin{smallmatrix} \textcolor{red}{c_{11}} \, c_{12} \\ c_{21} \, c_{22} \end{smallmatrix}\right]\)\newline
\textbf{\textcolor{red}{!! each row * each column -> length must be the same}}\newline

\subsection{Matrix Transpose}
\(\left[\begin{smallmatrix} 1  5 \\ 6  7 \end{smallmatrix}\right]^T == \left[ \begin{smallmatrix} 1  6 \\ 5  7 \end{smallmatrix}\right]\)\newline
\textcolor{purple}{Just swap row and columns}

\subsection{Inverse of a Matrix}
\(A^{-1}A = I\)\newline
\includegraphics[scale=0.2]{2022-10-05-05:07:54.png}

\subsection{Linear Dependence}
if the addition of 2 vectors with a scalar of NOT 0 is not the null-vector, then they are linearly \textbf{independent}.\newline
They are also linearly independent if \textbf{the determinant is NOT 0, or the gauss algorithm ends in unitmatrix}.

\subsection{Length of a vector}
\textcolor{purple}{\( |\vec{v}| = \sqrt{x^2 + y^2 + z^2} \)}

\subsection{Dotproduct}
\textbf{\textcolor{red}{\( \vec{a} * \vec{b} = |\vec{a}| * |\vec{b}| * cos(\alpha) \)}}\newline
returns 0 when angle is 90

\subsection{Crossproduct}
\textbf{\textcolor{red}{\( \vec{a} \text{ x } \vec{b} = |\vec{a}| * |\vec{b}| * sin(\alpha) * n\)}}\newline
This is usually used to get the unit vector of the resulting vector, or to\newline
calculate the area of a parallelogram\newline
\(\alpha\) = angle between the vector \(\vec{a}\) and \(\vec{b}\)\newline
\includegraphics[scale=0.4]{2022-12-27:07:39:36.png}

\subsection{Coordinate systems/spaces}
right handed: positive x to right\newline
left handed: positive x to left\newline
\textcolor{orange}{Local: world without transforms}\newline
\textcolor{orange}{World:} actual 3D representation \newline
Coord: engine defined: ex right handed\newline 
\textcolor{orange}{View:} view transformation for 2D \newline
Coord: engine, 180 rotated\newline
\textcolor{orange}{Clip:} creates 3D illusion\newline
\textcolor{orange}{Screen:} endresult for the human\newline
Coord: without near-far axis (z)

\subsection{2D Projection}
camera: (\(e_x, e_y, e_z\)), P: (x,y,z)\newline
\minipg{
  \textcolor{purple}{\(y_{\text{new}} = \dfrac{e_y z - e_z y}{z - e_z}\)}\newline
  \textcolor{purple}{\(x_{\text{new}} = \dfrac{e_x z - e_z x}{z - e_z}\)}\newline
  \textcolor{purple}{\( y = \dfrac{\Delta y}{\Delta z} z + c\)}\newline
}{
\includegraphics[scale=0.3]{2022-09-21-03:39:39.png}\newline
}[0.12,0.1]\newline
Note, 2D is usually done with \textcolor{red}{\textbf{homogenous coordinates}}, which are just 3D coordinates with a fixed z axis, eg. 1

\subsection{Bresenham Line Algorithm}
\textcolor{purple}{\(y = \dfrac{\Delta y}{\Delta x}x + c = mx + c\) ==> \(m = \dfrac{\Delta y}{\Delta x}\)}\newline
\(x = x_0 + \Delta x\) \(y = y_0 + m\Delta x\) \newline \(\Delta x = x - x_0\) \(\Delta y = y - y_0 \)

\section{OpenGL}
 \textcolor{purple}{State Machine, Context with ID, \newline Method based, Originally for c++}
\begin{itemize}
\item \textcolor{orange}{uniform:} available everywhere
\item \textcolor{orange}{in:} input to shader
\item \textcolor{orange}{out: } output from shader
\end{itemize} 

\subsection{Meshes}
\textcolor{purple}{Points(x,y,z), Edges, Faces(areas), consisting of triangles}
Vertices can either be indexed, or non-indexed.\newline
\textcolor{orange}{Non-index: 1 array, duplicate points}\newline
\textcolor{orange}{Indexed: 2 arrays(vertices and triangles), no duplicates!}\newline
\begin{lstlisting}
// snipped getVertices
for(int j=0; j < sect; j++) {
vert[i++] = Cos(angle);
vert[i++] = Sin(angle);
vert[i++] = 0;
angle += DeltaAngle; }
// snipped getTriangles
for(int j=0; j < sect; j++) {
tria[i++] = 0;
tria[i++] = 1 + j;
tria[i++] = 1+(j+1) % sect);}
\end{lstlisting}

\subsection{Perspective Projection}
Similar to 2D Projection. \newline
\textcolor{orange}{View Frustum}: \newline
\textbf{fov(y), aspectratio, near and far plane}\newline
\textcolor{OliveGreen}{If the fov is small, then shapes appear big -> fov is big, shapes appear small}

\subsection{Orthographic Projection}
\textcolor{purple}{normalized coordinates (between -1 and 1)} \textbf{No Z dimension!}

\section{GPU Pipeline | Shaders}
\begin{itemize}
\item \textcolor{purple}{Vertex Processor} Programmable\newline
  Turns \textbf{raw vertices} into primitives
\item \textcolor{purple}{Rasterizer} primitives into fragments
\item \textcolor{purple}{Fragment Processor} Programmable \newline
  processes fragments -> ex. \textbf{pixels}
\item \textcolor{purple}{Output Merging} Fragments to pixels
\end{itemize}
Note a shader would program both the vertex and the fragment processor!

\subsection{Double Buffering}
This seperates the writing and reading of frames, making sure there are no artifacts when displaying the frame, see tearing!

\section{Lighting}
\begin{itemize}
\item \textcolor{orange}{Ambient Lighting}\newline
\textcolor{purple}{From all directions, Remission in all directions, camera independent}\newline
\item \textcolor{orange}{Diffuse Lighting}\newline
\textcolor{purple}{point light, remission in all directions}\newline
Remission proportional to cosine between normal and light vector\newline
\minipg{
\textcolor{OliveGreen}{For matt surfaces}\newline
Depends: Camera, normals, lightpos
}{
\includegraphics[scale=0.2]{2022-12-28:11:46:29.png}
}[0.12,0.1]
\item \textcolor{orange}{Specular Lighting}\newline
\textcolor{purple}{point light, remission in \textbf{one} direction}\newline
Intensity dependent on angle between reflection vector and camera angle\newline
\textcolor{OliveGreen}{for reflective surfaces}\newline
\includegraphics[scale=0.2]{2022-12-28:11:46:39.png}
\item \textcolor{orange}{Blinn-Shading}\newline
  \textcolor{purple}{Combination of ambient, diffuse and specular}\newline
  if angle over 90degrees, can't display reflections
\item \textcolor{orange}{Blinn-Phong-Shading}\newline
  \textcolor{purple}{Modification of Blinn, angle between halfway vector and normals}\newline
  \minipg{
    Depends: Camera, lightpos, normals, fragmentpos 
  }{ 
  \includegraphics[scale=0.2]{2022-12-28:11:52:11.png}
  }[0.12,0.1]
\end{itemize} 

\subsection{Subtractive Light (CMYK)}
\textcolor{purple}{Start with white, then remove colors}\newline
When an object remits light, it will only show colors that are both in the light AND on the object itself:\newline
\textcolor{purple}{\(C = C_L^T * C_0\)}\newline
\(C_0\) is the original color\newline
\textcolor{red}{The base Cyan, Magenta, Yellow are stored in RGB vectors!}

\subsection{Additive Lighting}
\textcolor{purple}{Start with black, then add colors}\newline
\textcolor{purple}{\(C = 1 - (1 - C_{L1} )^T (1 - C_{L2} )\)}

\subsection{Representation}
\textcolor{purple}{Quantity of primitives} polygons\newline
\textcolor{purple}{approximation} splines, NURBS\newline
\textcolor{purple}{constructed} subdivision surfaces

\subsection{Sweep Triangulation}
\includegraphics[scale=0.25]{2022-10-19-03:21:54.png}\includegraphics[scale=0.25]{2022-10-19-03:22:00.png}\newline

\subsection{Insert Triangulation}
\includegraphics[scale=0.25]{2022-10-19-03:28:11.png}
\includegraphics[scale=0.25]{2022-10-19-03:36:45.png}\newline
After Insert or sweep, swap long edge with shorter edge for better performance\newline

\subsection{Subdivision Surfaces}
\begin{itemize}
\item \textcolor{purple}{Recursive refinement}\newline
starts blocky, gets smooth over iterations\newline
\textcolor{OliveGreen}{Combination of meshes and splines}
\item \textcolor{purple}{Chaikings Algorithm}\newline
\minipg{
\includegraphics[scale=0.35]{2022-12-28:02:16:51.png}
}{
Replace point in the middle with 2 new points, with 3/4 of previous and 1/4 of next point}[0.135,0.1]
\item \textcolor{purple}{Triangle Algorithms}\newline
  Loop, \(\sqrt{3}\) Subdivision
\item \textcolor{purple}{Rectangle Algorithms}\newline
  Catmul-Clark, Doo-Sabin
\end{itemize}

\subsection{Model Correction}
Rugular nodes: 6 edges for inner nodes, 4 or 3 edges for outer nodes\newline
\textcolor{red}{irregular edges are "corrected"}
\minipg{
\textcolor{teal}{Area based:}\newline
\textcolor{green}{ Surface characteristics stay}\newline 
\textcolor{red}{ inferior detection and solving of errors }\newline 
}{
\textcolor{teal}{Volume based:}\newline
\textcolor{green}{ Good detection and solving of errors}\newline 
\textcolor{red}{ often too detailed, subpar triangulation quality\newline
surface characteristics are lost} 
}[0.1,0.15]

\subsection{Check Quality of Meshes}
\textcolor{purple}{Inspection of Reflectionlines}\newline
\textcolor{purple}{Triangle Form}

\subsection{Mesh reduction}
Recude the amount of vertices while still getting the same model
\begin{itemize}
\item \textcolor{purple}{Vertex Clustering (least squares)}
\item \textcolor{purple}{Incremental mesh reduction}\newline
\includegraphics[scale=0.2]{2022-12-28:04:47:14.png}
\includegraphics[scale=0.2]{2022-12-28:04:47:17.png}
\includegraphics[scale=0.2]{2022-12-28:04:49:44.png}
\includegraphics[scale=0.2]{2022-12-28:04:49:47.png}
\item \textcolor{purple}{Remeshing}\newline
new mesh with less triangles
\end{itemize} 

\subsection{Rasterization/Aliasing}
\minipg{
\textbf{Z-Buffer}\newline
higher values: further away\newline
lower values: closer\newline
}{
\includegraphics[scale=0.3]{2022-12-28:04:54:31.png}
}[0.1,0.1]
\textcolor{purple}{Z-Fighting}: \newline
Objects with same z value fight over who is infront -> flickering

\subsection{Anti-Aliasing/Super-sampling}
\minipg{
simple variant, more points -> less transparent\newline
\textcolor{red}{Not for textures, as it only runs once}
}{
\includegraphics[scale=0.3]{2022-12-28:04:59:42.png}
}[0.12,0.1]

\subsection{Mipmaps}
texture resolution dynamic to distance to camera.\newline
removes aliasing and solves textures with too little detail

\section{Reflections}
static boundary | texture boundary\newline
\includegraphics[scale=0.3]{2022-12-28:05:47:49.png}\includegraphics[scale=0.3]{2022-12-28:05:47:55.png}

\section{Shadow mapping}
Projection of light to surface with depth as the "color" -> Z buffer\newline
\textcolor{red}{For textures it is better to save the position and color to make shadows easier}

\subsection{HDRI}
\textcolor{purple}{Take 3 different pictures with differnt lighting and put them together\newline
\textbf{day, night, sunset}}

\section{Ray Tracing}
\minipg{
\includegraphics[scale=0.15]{2022-12-28:06:03:43.png}
}{
\textcolor{red}{Denoising: Enables Real-time Ray Tracing with deep-learning}
}[0.17,0.08]

\subsection{Acceleration Structures}
\minipg{
\textcolor{purple}{Bounding Volumes Hierarchy}\newline
These are simple volumes around a texture that will \textbf{hierachically} be checked to see which texture will be hit by ray.\newline
}{
\includegraphics[scale=0.3]{2022-12-28:06:25:23.png}
}[0.1,0.12]

\subsection{Quad-Tree}
\minipg{
\textcolor{purple}{Recursively divide square if more than 1 point in square}\newline
Oct-tree -> same for 3D
}{
  \includegraphics[scale=0.2]{2022-12-28:06:14:23.png}
}[0.1,0.12]

\section{webGL}
\begin{itemize}
\item \textcolor{purple}{Scene graph treejs}
\item \textcolor{purple}{Canvas in html}
\item \textcolor{purple}{Single threaded!}
\item \textcolor{purple}{many primitives predefined}
\end{itemize} 


\end{multicols*}
\end{document}
