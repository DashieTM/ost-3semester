\documentclass[main.tex,fontsize=8pt,paper=a4,paper=portrait,DIV=calc,]{scrartcl}
\input{../template-exam.tex}
\begin{document}
\section{Terms and Definitions}
\begin{table}[h!]
\begin{tabular}{|p{0.2\linewidth}|p{0.755\linewidth}|}
\hline

\hline
\hline
\hline
\end{tabular}
\end{table}
\section{General Rules}
All physical rules are universal, this means there is no space in the universe where these rules do not apply. \\ In other words, shit like equestria is sadly not possible according to physics, what a surprise!
\section{Vectors}
\textbf{Notation} \( \vec{A} = \begin{bmatrix}x_{1} \\ x_{2} \end{bmatrix} \) OR \(\vec{A} = (x1 | x2)\)
\,   \,As you have learned in vocational school, use the first for vectors, the second for points.\\
\pictext{2022-09-23-10:41:26.png}
{A vector is nothing but a projection in the X,Y,Z planes \\ 
It has both a direction and a value\\
This value is usually modified by a scalar aka a factor. Then the vector can be something more universal \\
We call this universal vector a unit vector. A directional vector with the value 1.}[0.6,0.35,0.3]\\
\pictext{2022-09-23-10:41:32.png}{}[0.6,0.35,0.3]\\
\pictext{2022-09-23-10:41:37.png}{If the direction and the value is the same, then there is nothing that differentiates this vector from another.}[0.6,0.35,0.3]\\
\pictext{2022-09-23-10:41:44.png}{A vector is identical if both the value and the direction is the same.}[0.6,0.35,0.3]\\
\pictext{2022-09-23-10:41:48.png}{Similar, if the value is identical but the direction is negated, then you have the inverse vector.}[0.6,0.35,0.3]\\
\pictext{2022-09-23-10:41:53.png}{simple vector math}[0.6,0.35,0.3]\\
\subsection{Dotproduct}
\Large{\textbf{\textcolor{red}{\( \vec{a} * \vec{b} = |\vec{a}| * |\vec{b}| * sin(\alpha) * n \)}}}\newline
\normalsize This returns a scalar, aka a number that you can use.\newline
This will return 0 when used in a right angle as \(cos(90)\) is 0!\\
\subsection{Crossproduct}
\Large{\textbf{\textcolor{red}{\( \vec{a} \text{ x } \vec{b} = |\vec{a}| * |\vec{b}| * cos(\alpha) \)}}}\newline
\normalsize This is usually used to get the unit vector of the resulting vector.\newline
\(\alpha\) is the angle between the vector \(\vec{a}\) and \(\vec{b}\)\newline
\emph{\textcolor{teal}{For more information and proof, check the dedicated vector Document from vocational school.}}\\
\pagebreak
\begin{multicols*}{2}
\section{Static}
\subsection{Forces}
\subsubsection{Types}
\vspace{2mm}
\begin{itemize}
\item \textcolor{purple}{movement -> accelleration and braking}
\item \textcolor{purple}{deformation}

\end{itemize}

\subsubsection{Gravity}
\textcolor{purple}{\( |\vec{F_G}| = G * \dfrac{m_1 * m_2}{r^2} \)} | 
G = \(6.67 * 10^{-11}\dfrac{m^3}{kgs^2}\)\vspace{2mm}

\subsubsection{Fall Accelleration}
\textcolor{purple}{\(|\vec{F_G}| = m * g\)} | with \(g = 9.81\dfrac{m}{s^2}\)

\subsubsection{Newton}
\textcolor{purple}{\(1N = 1kg * \dfrac{m}{s^2}\)}\vspace{2mm}

\subsection{MassPoint}
\subsubsection{Term Definition} 
A Masspoint is simply a weight without a volume, aka it is the center of mass itself.

\subsubsection{regular friction}
regular friction has 2 formulas. \newline
standing friction = standing coefficient * force of mass \newline
\large \textcolor{red}{\(F_{s} = \gamma_s * F_G\)}\newline 
\normalsize friction = moving coefficient * force of mass \newline 
\large\textcolor{red}{\(F_{m} = \gamma_m * F_G\)} \normalsize\newline
\pic{2022-09-30-11:13:56.png}

\subsubsection{viscose friction}
Viscose friction happens when you do not have a rigid/dry body.\newline
A good example for this are tires. They act as a "lubricant" in proper conditions -> see F1.\newline
In this case the more contact you have to the floor, the less friction you will have.\newline
Interestingly this also means that it is very unlikely that you get stuck in mud/snow with small tires,\newline
but very much likely that you do get stuck with big slick tires -> also see f1 kekw

\subsubsection{Actio \& Reactio}
A force being inflicted upon a body will be met with the same force by said body.\newline
The only difference here is that the body with more mass has more "lag". \newline
In other words, more force is needed to accelerate that body.\newline
This means that the only reason why the earth doesn't follow you when you jump,\newline
is that the earth is by magnitudes heavier than you!\newline
\pic{2022-09-30-11:28:18.png}

\section{torque}
\subsubsection{line of effect}
A vector of a force can be used anywhere on that vector.\newline
This means that the point of effect can be moved infinitely in the vector line.\newline
\includegraphics[scale=0.2]{2022-09-30-11:37:46.png}

\subsubsection{lever} 
\minipg{
}{\includegraphics[scale=0.2]{2022-09-30-11:44:32.png}}[0.2,0.2]

\subsubsection{Force and Acceleration} 
\textcolor{teal}{A force will always result in an acceleration, this can either be positive or negative(braking).\newline
However remember that an acceleration of 0 does not mean braking, it means constant speed!}

\subsubsection{torque}
\Large \( |\vec{M}| = a * |\vec{F}| \)\newline
\, \newline
\textcolor{teal}{ \normalsize \textbf{The value of Torque is always the length * force.}\newline
And it does not matter what constellation of force and length you use\newline
It will always be the same result as when the force increases with a new constellation the length decreases!}\newline
\, \newline
\textbf{Torque in vector format}\newline
It is simply the cross product of 2 vectors!\newline
\pic{2022-10-07-10:47:00.png}\newline
\minipg{
\Large \textcolor{teal}{These 2 can be combined!}\newline
\normalsize \textcolor{teal}{Notice the length and the force F, they have a 90 degree angle as is needed.\newline
\textbf{However the same calculation can be achieved by using the value of the cross product between r and F.\newline
This can be done as long as we have a reference point, here R!.}}\newline
\, \newline}
{\pic{2022-10-07-10:50:19.png}}[0.3,0.2]
\Large if reference point available:\newline \( |\vec{M}| = | \vec{r} \text{x} \vec{F} | = a * | \vec{F} | \ \)

\section{Mass and Equilibrium}

\subsubsection{Center of Mass} 
\minipg{
\textbf{\textcolor{teal}{ The sum of a mass * length -> level over the sum of all masses will result in either the center of mass,\newline
or an axis of it -> x, y, z.}}
}
{
\, \newline
\Large \( x_s = \dfrac{\sum_i x_i * m_i }{\sum_i m_i} \)\newline
\, \newline
\Large \( y_s = \dfrac{\sum_i y_i * m_i }{\sum_i m_i} \)\newline
\, \newline
\Large \( z_s = \dfrac{\sum_i z_i * m_i }{\sum_i m_i} \)\newline}[0.3,0.2]

\subsubsection{Equilibrium}
If you want an equilibrium, then all forces need to combine into 0!\newline
\, \newline
\Large \textcolor{teal}{\( \displaystyle\sum_{n}^{i=1}\vec{F_i} = \vec{0} \) AND \( \displaystyle\sum_{m}^{i=1} \vec{M_i} = \vec{0} \)}\newline
\normalsize 

\subsubsection{Types of Equilibriums}
\begin{itemize}
\item \textcolor{purple}{stable equilibrium}\newline
pushing an item in this state will cause the item to return to the original form.
\item \textcolor{purple}{unstable equilibrium}\newline
Pushing an item in this state will destroy the equilibrium.
\item \textcolor{purple}{indifferent equilibrium}\newline
Pushing an item in this state will create a new equilibrium.

\end{itemize} 

\subsection{Tension, Tortion, Contraction, Compression}
\subsubsection{Pullpressure} 
\textcolor{teal}{The pullpressure is the vertical pressure per area.}\newline
\, \newline
\large \( \sigma = \dfrac{F}{A} \)\newline
\normalsize 

\subsubsection{Pushpressure} 
\textcolor{teal}{the pushpressure is the parallel pressure per area.}\newline
\, \newline
\large \( \tau = \dfrac{F}{A} \)\newline
\normalsize 

\subsubsection{Stretching} 
\, \newline
The stretching is \textbf{similar} to the pullpressure.\newline
This can be visualized via this formula:\newline
\, \newline
\large \( \epsilon = \dfrac{\Delta l}{l} \approx \dfrac{F}{A} = \sigma \) | | | \large \( \epsilon = \dfrac{1}{E} * \sigma \)
\, \newline
\normalsize 

\subsubsection{crosscontraction}
\, \newline
\large \( \epsilon_q = \dfrac{\Delta d}{d} \)\newline
\, \newline
\normalsize 

\subsubsection{crosscontraction as Poisson-Number}
\, \newline
\large \( \epsilon_q = -\gamma \epsilon \)\newline
\, \newline
\normalsize 

\subsubsection{Compressability}
\, \newline
\large \( \dfrac{\Delta V}{V} = - \kappa * \Delta p  \)
\, \newline
\, \newline
\large \( \kappa = 0 \text{ or } \infty \text{ -> } \gamma = 0.5  \)
\, \newline
\normalsize 

\section{kinematic}
\subsection{Median Accelleration}
\textcolor{orange}{The median acceleration is the end velocity - the start velocity over the end time - the start time:}\newline
\, \newline
\large \( \overset{\_}{a} = \dfrac{v_2 - v_1}{t_2 - t_1} = \dfrac{v(t_2) - v(t_1)}{t_2 - t_1} \) \newline
\normalsize 

\subsubsection{Current Acceleration}
\textcolor{orange}{The current acceleration is the limit of the current velocity - the velocity of current time - delta time over delta time. \newline
Or easier, the derivation of current velocity:}\newline
\, \newline
\large \( a(t) = \underset{\Delta t -> 0}{\lim} \dfrac{v(t) - v(t - \Delta t)}{\Delta t} = \dfrac{d}{dx}v(t)\) \newline
\normalsize

\subsubsection{Derivatives}
\, \newline
\large \(\dfrac{d}{dt}a(t) = v(t) = \int{s(t)} \)\newline
\, \newline
\large \(\dfrac{d}{dt}v(t) = s(t)\)\newline
\, \newline
\normalsize \textcolor{orange}{Also note that when you integrate, you get a constant c -> hence the following:}\newline
\, \newline
\large \( v(t) = \int{a(t) dt} = \int{0 dt} = c \text{ --> if a(t) == 0}\) \newline
\, \newline
\normalsize \textcolor{red}{This is why the acceleration on earth -> gravity is constant!! HOLY FUCK}

\subsubsection{Derivative of a Vector}
\textcolor{orange}{The derivative of a vector is the derivative of each component, meaning the following:}\newline
\large \(\dfrac{d}{dx} \vec{r}(t) = (\dfrac{d}{dx} x(t), \dfrac{d}{dx} y(t), \dfrac{d}{dx} z(t) ) \)\newline
\, \newline \normalsize

\subsubsection{Current Velocity with Vectors}
\textcolor{orange}{This specifies the current velocity of a vector:}\newline
\, \newline
\large \( \vec{v}(t) = \underset{\Delta t -> 0}{\lim} \dfrac{\Delta \vec{r}}{\Delta t} = \dfrac{d}{dx} \vec{r} = \overset{.}{\vec{r}} \) \newline 
\, \newline \normalsize

\subsubsection{Radial Acceleration}
\, \newline
\large \( a_{tan} = \underset{\Delta t -> 0}{lim} \dfrac{\Delta v_{tan}}{\Delta t} = \dfrac{dv}{dt} = \overset{.}{v} \) \newline
\, \newline
\large \( \dfrac{\Delta v_{rad}}{v} = \dfrac{\Delta r}{r}\)
\, \newline
\large \( a_{rad} = \underset{\Delta t -> 0}{\lim} \dfrac{\Delta v_{rad}}{\Delta t} = \dfrac{v}{r}\underset{\Delta t -> 0}{\lim} \dfrac{\Delta v}{\Delta t} = \textcolor{red}{\dfrac{v^2}{r}} \) \newline
\, \newline
\large \( v(t) = a_{tan} * t + v_0 \) \newline
\large \( s(t) = \dfrac{a_{tan} * t^2}{2} + v_0 * t + s_0\) \newline
\, \newline \normalsize \
\textcolor{orange}{In case that \(a_{tan}\) is 0, you obviously only evaluate \(v_0 * t + s_0\) , this is because the first part would evaluate to 0.}

\subsubsection{Angle Velocity}
\, \newline
\large \( \omega = \underset{\Delta t -> 0}{\lim} \dfrac{\phi(t + \Delta t - \phi(t))}{\Delta t} = \dfrac{d\phi}{dt} = \overset{.}{\phi} \) \newline
\, \newline \normalsize
\begin{itemize}
\item \textcolor{orange}{\(\phi\) == angle}
\item \textcolor{orange}{\(\omega\) == angle acceleration}
\item \textcolor{orange}{ t == time}

\end{itemize}

\subsubsection{Frequency and Circulation}
\, \newline
\large \( f = \dfrac{1}{T} <=> T = \dfrac{1}{f} \)\newline
\, \newline
\large \( \omega = \dfrac{2\pi}{T} <=> T = \dfrac{2\pi}{\omega} \)\newline
\, \newline
\large \( \omega = 2\pi f <=> f = \dfrac{\omega}{2\pi} \)\newline
\, \newline
\large \textcolor{orange}{\( r \omega = \dfrac{2\pi r}{T} = v \)}\newline
\, \newline
\large \textcolor{orange}{\( \alpha = \underset{\Delta t -> 0}{\lim} \dfrac{\omega (t + \Delta t) - \omega (t)}{\Delta t} = \dfrac{d\omega}{dt} \)}\newline 
\, \newline
\large \textcolor{orange}{\( \alpha_{tan} = \dfrac{dv}{dt} = \dfrac{d}{dt}r * \omega = r * \alpha \)}\newline 
\, \newline
\textcolor{purple}{constant angle-motion (no acceleration)}\newline
\large \textcolor{orange}{\( \Phi(t) = \omega_0 * t + \Phi_0 \)  | given \(\alpha == 0 \) and \( \omega = \omega_0 == \text{ constant} \)}\newline
\, \newline
\textcolor{purple}{constant accelerated angle-motion}\newline
\large \textcolor{orange}{\( \Phi(t) = \dfrac{\alpha_0 t^2}{2} + \omega_0 * t + \Phi_0 \) \newline given \( \alpha == \alpha_0 == \text{ constant} \) and \(\omega == \alpha_0 * t + \omega_0 \) }\newline
\, \newline \normalsize
\begin{itemize}
\item \textcolor{orange}{T = Time per Circulation}
\item \textcolor{orange}{\(f\) == frequency}
\item \textcolor{orange}{r == radius}
\item \textcolor{orange}{v == velocity}
\item \textcolor{orange}{\(\Phi\) == angle}
\item \textcolor{orange}{\(2\pi\) == 1 Circulation}
\end{itemize}

\subsection{Free Fall and Parables}
\subsubsection{Vertical Throw}
\, \newline
\large \textcolor{orange}{\( \alpha = -g == \text{ constant}  \)}\newline
\, \newline
\large \textcolor{orange}{\( v(t) = -g * t + v_0 \)}\newline
\, \newline
\large \textcolor{orange}{\( h(t) = \dfrac{-g * t^2}{2} + v_o * t + h_0  \)}\newline
\, \newline
\large \textcolor{purple}{\( t_{max} = \dfrac{v_0}{g} \)}\newline
\, \newline
\large \textcolor{purple}{\( h_{max} = h(t_{max}) = \dfrac{-g * v^{2}_{0}}{2 g^2 } + \dfrac{v_{0}^{2}}{g} + h_0 = \dfrac{v_{0}^{2}}{2g} + h_0 \)}\newline
\, \newline \normalsize
\begin{itemize}
\item \textcolor{orange}{a == acceleration}
\item \textcolor{orange}{v = velocity}
\item \textcolor{orange}{h = height}
\item \textcolor{orange}{t = time}
\item \textcolor{orange}{g = gravity -> 9.81}
\end{itemize} 

\subsubsection{Free Fall}
\, \newline
\large \textcolor{orange}{\( a = -g == \text{ constant} \)}\newline
\, \newline
\large \textcolor{orange}{\(v(t) = -g * t\)}\newline
\, \newline
\large \textcolor{orange}{\(h(t) = \dfrac{-g * t^2}{2} + h_0\)}\newline
\, \newline \normalsize
\begin{itemize}
\item \textcolor{orange}{a == acceleration}
\item \textcolor{orange}{v = velocity}
\item \textcolor{orange}{h = height}
\item \textcolor{orange}{t = time}
\item \textcolor{orange}{g = gravity -> 9.81}
\end{itemize} 

\subsubsection{Throw Distance}

\subsection{Impulses and Impulsretention}

\subsubsection{Impulse}
\, \newline
\large \textcolor{purple}{\( \vec{p} = m\vec{v} \)}\newline
\, \newline
\textcolor{purple}{\( \vec{F} = m\vec{d} = m * \dfrac{d\vec{v}}{dt} = \dfrac{d}{dt} * (m\vec{v}) = \dfrac{d\vec{p}}{dt} \)}\newline
\normalsize \, \newline

\subsubsection{Force Push (kekw-wars)}
\, \newline
\large \textcolor{purple}{ \(\int^{t+\delta t}_{t} F(t) dt = \vec{F}* \delta t = \delta p\)}\newline
\normalsize \, \newline

\subsubsection{Impulsretention}
\, \newline
\large \textcolor{purple}{ \( \vec{p} = \int \dfrac{dp}{dt} dt = c \text{ => constant}\)}\newline
\Large \, \newline
Note, \(\dfrac{dp}{dt}\) is 0, therefore in a completely isolated system, the impulse retention is always \textbf{constant!} \normalsize \newline


\subsection{Center of Mass and Impulses} 
\subsubsection{Center of Mass in a collision}
\vspace{2mm}
\large \textcolor{purple}{\( m_1 * \vec{r_1} + m_2 * \vec{r_2} = \vec{0} \text{ <=> } m_1*\vec{v_1} + m_2 * \vec{v_2} = \vec{0} \)}\newline
\normalsize \, \newline
Types of collisions:\newline
\begin{itemize}
\item \textcolor{orange}{straight: the vectors of the center of mass are on one straight line}
\item \textcolor{orange}{tilted: The vectos of the center of mass complete an angle}
\item \textcolor{orange}{central: The center of mass of each collision entity is on the collision normal}
\item \textcolor{orange}{excentric: The center of mass of each collision entity is NOT on the collision normal }
\item \textcolor{orange}{elastic: The total kinetic energy before and ater the collision are the same}
\item \textcolor{orange}{inelastic: The total kinetic energy before and after the collision are different}
\item \textcolor{orange}{completely inelastic: The collision entities move at the same speed after the collision}
\end{itemize} 

\subsubsection{Deformation"Work"}
\vspace{2mm}
\large \textcolor{purple}{\( Q = (E_1 + E_2) - (E_1' + E_2') \geq 0 \)}\newline
\normalsize \, \newline
Legend:\newline
\begin{itemize}
\item \textcolor{orange}{\( E_1 + E_2 =\) Energy before collision}
\item \textcolor{orange}{\( E_1' + E_2' =\) Energy after collision}
\item \textcolor{orange}{Q = DeformationWork}
\end{itemize} 

\subsubsection{Elasticity Count}
\vspace{2mm}
\large \textcolor{purple}{\( k = \dfrac{v_2' - v_1'}{v_1 - v_2} \)}\newline
\normalsize \, \newline
Legend:\newline
\begin{itemize}
\item \textcolor{orange}{\(v_1\) = velocity of entity 1}
\item \textcolor{orange}{\(v_2\) = velocity of entity 2}
\item \textcolor{orange}{k = elasticity count}
\end{itemize} 


\subsection{straight, central, completely elastic collision}

\subsubsection{Impulse Law}
\vspace{2mm}
\large \textcolor{purple}{\( p \overset{!}{=} p' \text{ <=> } m_1v_1 + m_2v_2 \overset{!}{=} m_1v_1' + m_2v_2' \)}\newline
\normalsize \, \newline
Legend:\newline
...

\subsubsection{Energy Law}
\vspace{2mm}
\large \textcolor{purple}{\( E_{\text{kin}} \overset{!}{=} E_{\text{kin}}' \text{ <=> } \dfrac{m_1v_1^2}{2} + \dfrac{m_2v_2^2}{2} \overset{1}{=} \dfrac{m_1v_1'^2}{2} + \dfrac{m_2v_2'^2}{2} \)}\newline
\normalsize \, \newline
Legend:\newline
...

\subsubsection{Special case with \(m_1 = m_2\)}
\vspace{2mm}
\large \textcolor{purple}{\( v_1' = 0 + 1 * 0 = 0 \)}\newline
\textcolor{purple}{\( v_2' = v_1 + 0 = v_1\)}\newline
\normalsize \, \newline
Legend: \newline
...


\subsection{Straight, central, completely inelastic collision}

\subsubsection{Impulse Law}
\vspace{2mm}
\large \textcolor{purple}{\( m_1v_1 = m_2v_2 = (m_1 + m_2 )v' \text{ => } v' = \dfrac{m_1v_1 + m_2v_2}{m_1 + m_2}\)}\newline
\textcolor{red}{\( p = p' \)!!}\newline
\normalsize \, \newline
Legend:\newline
...

\subsubsection{Energy Law}
\vspace{2mm}
\large \textcolor{purple}{\( \dfrac{m_1v_1^2}{2} + \dfrac{m_2v_2^2}{2} = \dfrac{(m_1 + m_2)v'^2}{2} + Q \)}\newline
\, \newline
\textcolor{purple}{\(Q = \dfrac{m_1m_2}{2(m_1+m_2)}*(v_1 -v_2)^2\)}\newline
\normalsize \, \newline
Legend:\newline
\begin{itemize}
\item \textcolor{orange}{\(m_1\) = Center of mass entity 1}
\item \textcolor{orange}{\(m_2\) = Center of mass entity 2}
\item \textcolor{orange}{\(v_1\) = velocity entity 1}
\item \textcolor{orange}{\(v_2\) = velocity entity 2}
\item \textcolor{orange}{Q = DeformationWork}
\end{itemize}


\subsection{Reduced Mass, Relative Velocity}
\subsubsection{}
\vspace{2mm}
\large \textcolor{purple}{\(v_{\text{rel}} = | v_1 - v_2|\)}\newline
\textcolor{purple}{\(\mu = \dfrac{m_1m_2}{m_1 + m_2}\)}\newline
\textcolor{purple}{\( Q = \dfrac{\mu v_{\text{rel}^2}}{2}\)}\newline
\normalsize \, \newline
Legend:\newline
\begin{itemize}
\item \textcolor{orange}{\(v_1\) = velocity entity 1}
\item \textcolor{orange}{\(v_2\) = velocity entity 2}
\item \textcolor{orange}{\(\mu\) = reduced mass}
\item \textcolor{orange}{Q = DeformationWork}
\item \textcolor{orange}{\(v_{\text{rel}}\) = relative velocity}
\end{itemize}



\end{multicols*}
\end{document}

