\documentclass[main.tex,fontsize=8pt,paper=a4,paper=portrait,DIV=calc,]{scrartcl}
% Document
\usepackage[T1]{fontenc}
\usepackage[dvipsnames]{xcolor}
\usepackage[nswissgerman,english]{babel}
\renewcommand{\familydefault}{\sfdefault}

% Format
\usepackage[top=5mm,bottom=1mm,left=5mm,right=5mm]{geometry}
%\setlength{\headheight}{\baselineskip}
%\setlength{\headsep}{0mm}

%\usepackage{scrlayer-scrpage}
%\clearpairofpagestyles
%\chead{{\bfseries\TITLE, \AUTHOR, \pagename~\thepage}}

%\addtokomafont{pagehead}{\upshape}

\usepackage{multicol}
\setlength{\columnsep}{2mm}
\setlength{\columnseprule}{0.1pt}

% Math
\usepackage{amsmath}
\usepackage{amssymb}
\usepackage{amsfonts}

% Code
\usepackage{fancyvrb, etoolbox, listings, xcolor}
%\usemintedstyle{bw}

%\newminted[shell]{bash}{
%fontsize=\footnotesize,
%fontfamily=tt,
%breaklines=true,
%frame=single,
%framerule=0.1pt,
%framesep=2mm,
%tabsize=2
%}
%\newminted{css}{
%breaklines=true,
%tabsize=4,
%autogobble=true,
%escapeinside=||,
%stripall=true,
%stripnl=true,
%}

    \definecolor{lightgray}{rgb}{0.95, 0.95, 0.95}
    \definecolor{darkgray}{rgb}{0.4, 0.4, 0.4}
    \definecolor{purple}{rgb}{0.65, 0.12, 0.82}
    \definecolor{ocherCode}{rgb}{1, 0.5, 0} % #FF7F00 -> rgb(239, 169, 0)
    \definecolor{blueCode}{rgb}{0, 0, 0.93} % #0000EE -> rgb(0, 0, 238)
    \definecolor{greenCode}{rgb}{0, 0.6, 0} % #009900 -> rgb(0, 153, 0)
    \definecolor{teal}{rgb}{0.0, 0.5, 0.5}

\lstdefinestyle{code}{
    identifierstyle=\color{black},
    keywordstyle=\color{blue}\bfseries\small,
    ndkeywordstyle=\color{greenCode}\bfseries\small,
    stringstyle=\color{ocherCode}\ttfamily\small,
    commentstyle=\color{teal}\ttfamily\textit\small,
    basicstyle=\ttfamily\small,
    breakatwhitespace=false,         
    breaklines=true,                 
    captionpos=b,                    
    keepspaces=true,                 
    showspaces=false,                
    showstringspaces=false,
    showtabs=false,                  
    tabsize=2,
    belowskip=-5pt
}



% Images
\usepackage{graphicx}
\newcommand{\pic}{\includegraphics[scale=0.3]}
\graphicspath{{Screenshots/}{../Screenshots}}
\makeatletter
\def\pictext#1#2{%
    \@ifnextchar[{%
    \pictext@iiiii{#1}{#2}%
    }{%
      \pictext@iiiii{#1}{#2}[0.5,0.4,0.3]% Default is 5
    }%
}
\def\pictext@iiiii#1#2[#3,#4,#5]{\begin{minipage}{#3\textwidth}\includegraphics[scale=#4]{#1}\end{minipage}\begin{minipage}{#5\textwidth}#2\end{minipage}}
\def\minipg#1#2{%
    \@ifnextchar[{%
    \minipg@iiii{#1}{#2}%
    }{%
      \minipg@iiii{#1}{#2}[0.3,0.6]% Default is 5
    }%
}
\def\minipg@iiii#1#2[#3,#4]{\vspace{0.8mm}\begin{minipage}{#3\textwidth}#1\end{minipage}\begin{minipage}{#4\textwidth}#2\end{minipage}{\vspace{0.8mm}}}
\makeatother

%\newenvironment{minty}[2]% environment name
%{% begin code
%  \begin{minipage}{#1}
%  \begin{minted}{#2}
%}%
%{% end code
%  \end{minted}
%  \end{minipage}
%  \end{minty}\ignorespacesafterend
%} 

% Smaller Lists
\usepackage{enumitem}
\setlist[itemize,enumerate]{leftmargin=3mm, labelindent=0mm, labelwidth=1mm, labelsep=1mm, nosep}
\setlist[description]{leftmargin=0mm, nosep}
\setlength{\parindent}{0cm}

% Smaller Titles
\usepackage[explicit]{titlesec}

%% Color Boxes
\newcommand{\sectioncolor}[1]{\colorbox{black!60}{\parbox{0.989\linewidth}{\color{white}#1}}}
\newcommand{\subsectioncolor}[1]{\colorbox{black!50}{\parbox{0.989\linewidth}{\color{white}#1}}}
\newcommand{\subsubsectioncolor}[1]{\colorbox{black!40}{\parbox{0.989\linewidth}{\color{white}#1}}}
\newcommand{\paragraphcolor}[1]{\colorbox{black!30}{\parbox{0.989\linewidth}{\color{white}#1}}}
\newcommand{\subparagraphcolor}[1]{\colorbox{black!20}{\parbox{0.989\linewidth}{\color{white}#1}}}

%% Title Format
\titleformat{\section}{\vspace{0.5mm}\bfseries}{}{0mm}{\sectioncolor{\thesection~#1}}[{\vspace{0.5mm}}]
\titleformat{\subsection}{\vspace{0.5mm}\bfseries}{}{0mm}{\subsectioncolor{\thesubsection~#1}}[{\vspace{0.5mm}}]
\titleformat{\subsubsection}{\vspace{0.5mm}\bfseries}{}{0mm}{\subsubsectioncolor{\thesubsubsection~#1}}[{\vspace{0.5mm}}]
\titleformat{\paragraph}{\vspace{0.5mm}\bfseries}{}{0mm}{\paragraphcolor{\theparagraph~#1}}[{\vspace{0.5mm}}]
\titleformat{\subparagraph}{\vspace{0.5mm}\bfseries}{}{0mm}{\subparagraphcolor{\thesubparagraph~#1}}[{\vspace{0.5mm}}]

%% Title Spacing
\titlespacing{\section}{0mm}{0mm}{0mm}
\titlespacing{\subsection}{0mm}{0mm}{0mm}
\titlespacing{\subsubsection}{0mm}{0mm}{0mm}
\titlespacing{\paragraph}{0mm}{0mm}{0mm}
\titlespacing{\subparagraph}{0mm}{0mm}{0mm}

%% format cells
\usepackage[document]{ragged2e}
\usepackage{array, makecell}
\renewcommand{\arraystretch}{2}
\newcommand{\mc}{\makecell[{{m{1\linewidth}}}]}



\begin{document}
\begin{table}[h!]
\section{Nix}
\begin{tabular}{|m{0.1\linewidth}|m{0.855\linewidth}|}
\hline
NixPkg & 
This is just another repository just like the arch repo.\newline
\textcolor{red}{The idea of nix is that if you were able to build it today, you should be able to build it 20 years from now, on any computer.}\newline
\begin{itemize}
\item \textcolor{purple}{Reproducible}
\item \textcolor{purple}{Cross Platform}
\item \textcolor{purple}{Reliability}
\vspace{-3mm}
\end{itemize} 
\\
\hline
NixOS & 
Linux distro based on the nix repository.\newline
Only stores packages in the nix store directory, while config files in the /etc directory are handled with syslinks to the nix directory.\newline
\\
\hline
Flakes & 
Experimental feature that gives you a developer shell as well as a build file.\newline
\textcolor{red}{The flake.nix file needs to be in the root directory of the project.}\newline
\begin{enumerate}
\item \textcolor{purple}{Download Nix}
\item \textcolor{purple}{mkdir -p ~/.config/nix
echo “experimental-features = nix-command flakes” > ~/.config/nix/nix.conf}
\item \textcolor{purple}{nix run nixpkgs\#hello}
\vspace{-3mm}
\end{enumerate} 
\\
\hline
Old default.nix and shell.nix &
Before the way to do nix packages was with default.nix for building and shell.nix for developer shell.\newline
Note that just like flake.nix, these files need to be in the root directory of  the project.\\
\hline
An example Flake & 
\vspace{2mm}
\begin{lstlisting}
{
  inputs.nixpkgs.url = github:nixos/nixpkgs;
  outputs = inputs:
    let system = "x86_64-linux";
      pkgs = inputs.nixpkgs.legacyPackages.${system};
    in {
      devShells.${system}.default = pkgs.mkShell {
        nativeBuildInputs = [ pkgs.fortune pkgs.cowsay ];
      };
    };
}
\end{lstlisting}
\, \newline
\textcolor{purple}{Cross-Platform Flake}
\begin{lstlisting}
{
  inputs.nixpkgs.url = github:nixos/nixpkgs;
  inputs.flake-utils.url = github:numtide/flake-utils;
  outputs = inputs: inputs.flake-utils.lib.eachDefaultSystem (system:
    let pkgs = inputs.nixpkgs.legacyPackages.${system};
    in {
      devShells.default = pkgs.mkShell {...};
    }
  );
}
\end{lstlisting}
\\
\hline
Creating a C project & 
\vspace{2mm}
\begin{lstlisting}
{
  inputs.nixpkgs.url = github:nixos/nixpkgs;
  inputs.flake-utils.url = github:numtide/flake-utils;
  outputs = inputs: inputs.flake-utils.lib.eachDefaultSystem (system:
    let pkgs = inputs.nixpkgs.legacyPackages.${system};
    in {
      devShells.default = pkgs.mkShell {
        buildInputs = [ pkgs.libtensorflow ];
      };
      packages.default = pkgs.stdenv.mkDerivation {
        pname = "hello";
        version = "0.0.1";
        src = ./.;
        makeFlags = [ "BINDIR=$(out)/bin" ];
        buildInputs = [ pkgs.libtensorflow ]; // this allows you to install a bunch of 
        dependencies on the fly, making sure every system can build this project,
        no more abando outdated fun!
      };
    }
  );
}
\end{lstlisting}
\, \newline
Please note that you still need the makefile, it's at the end still C, nix can'tdo magic.
\\
\hline
Nix vs Docker & 
In comparison to docker, nix \textbf{promises reproducability}, as it stays at the version you are at, unlike docker which might change versions the next second you run it again!\newline
Also, \textbf{docker follows the multiple desks paradigm}, which means that you have multiple spaces to work in which each have their own dependencies and tools.\newline
Nix however, \textbf{follows the one desk paradigm}, instead of creating multiple desks, it has blueprints that will be opened whenever you want to do something, meaning that your desk is always clean, switching is fast and easy and you can always rely on it.\newline
\textcolor{purple}{In fact, you can also create a docker image out of nix.... lel}
\\
\hline
\end{tabular}
\end{table}
\end{document}
