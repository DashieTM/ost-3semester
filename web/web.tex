\documentclass[main.tex,fontsize=8pt,paper=a4,paper=portrait,DIV=calc,]{scrartcl}
\input{../ost-summary-template.tex}

\lstdefinelanguage{JavaScript}{
  keywords={break, case, catch, continue, debugger, default, delete, do, else, false, finally, for, function, if, in, instanceof, new, null, return, switch, this, throw, true, try, typeof, var, void, while, with},
  morecomment=[l]{//},
  morecomment=[s]{/*}{*/},
  morestring=[b]',
  morestring=[b]",
  ndkeywords={class, export, boolean, throw, implements, import, this},
  keywordstyle=\color{blue}\bfseries,
  ndkeywordstyle=\color{darkgray}\bfseries,
  identifierstyle=\color{black},
  commentstyle=\color{purple}\ttfamily,
  stringstyle=\color{red}\ttfamily,
  sensitive=true
}

%%%%%define html as viable language
\lstset{
    language=JavaScript,
    style=code,
}
%%%%%

\begin{document}
\begin{table}[h!]
\section{Basics}
\begin{tabular}{|m {0.205\linewidth}|m{0.750\linewidth}|}
\hline
ECMAScript & 
Javascript is only an implementation of ECMAScript. So ECMA is the actual standard.\\
\hline 
debug without IDE & 
\begin{lstlisting}
debugger;
\end{lstlisting}
\, \newline
\textcolor{teal}{This simply makes the browser stop at this line if the dev tools are open!}\\
\hline
Facts about javascript &
\vspace{2mm}
\begin{itemize}
  \item \textcolor{teal}{It's dynamic}
  \item \textcolor{teal}{It's dynamically typed! (like python)}
  \item \textcolor{teal}{both functional and object-oriented}
  \item \textcolor{teal}{It fails silently --> "geil"}
  \item \textcolor{teal}{It's deployed as source code, so everything is at least source available}
  \item \textcolor{teal}{It's part of the web platform}
\end{itemize}
\\
\hline
\textbf{Primitives vs Objects} & 
\large\textcolor{red}{Primitives}
\normalsize
\begin{itemize}
  \item \textcolor{teal}{string, number, boolean, undefined}
  \item \textcolor{teal}{compared by value}
  \item \textcolor{teal}{always immutable}
\end{itemize}
\, \newline
\large\textcolor{red}{Objects}
\normalsize
\begin{itemize}
  \item \textcolor{teal}{Arrays, Regular Expressions, Functions}
  \item \textcolor{teal}{compared by Reference}
  \item \textcolor{teal}{Mutable by default}
  \vspace{-3mm}
\end{itemize}
\\
\hline
Types & 
\includegraphics[scale=0.45]{2022-10-11-10:33:44.png}\\
\hline
Boolean Checks & 
\textcolor{red}{Every value can be converted to a boolean}\newline
\begin{itemize}
  \item !!(null) => false
  \item Boolean( null ) => false
\end{itemize}
\, \newline
\begin{itemize}
  \item \textcolor{teal}{Logical Operators \&\& ||}
  \item \textcolor{teal}{Not !}
  \item \textcolor{teal}{Equality Checks === !== > >= < <=} true or false
  \item \textcolor{teal}{Value Equality Check == !=} true false or NaN -> cast to number
  \vspace{-3mm}
\end{itemize}
\, \newline
\minipg{
False Values:\newline
\begin{itemize}
  \item false
  \item 0 
  \item "" 
  \item null 
  \item undefined 
  \item NaN
\end{itemize}
}
{True Values:
\begin{itemize}
  \item "0" \textcolor{teal}{any string that has something in it is considered true}
  \item "false"
  \item \char`\[ \char`\]
  \item \char`\{ \char`\}
  \, \newline
  \, \newline
  \, \newline
\end{itemize}}\\
\hline 
NaN &
\vspace{2mm}
\begin{itemize}
  \item \textcolor{red}{is an error value} 
  \item 0 / 0 = NaN
  \item is of type number
  \item NaN == NaN is false \textcolor{teal}{// use isNan\char`\( \char`\) instead}
  \vspace{-3mm}
\end{itemize}\\
\hline
Infinity & 
Infinity is a number in js, can be used to represent the mathematical infinity.\newline
Will also be used if the number is too big!\\
\hline
Numbers & 
\textcolor{red}{Every value can be casted to a number!}\newline
\begin{itemize}
  \item +(true) = 1
  \item Number(true) = 1
  \item Number(null) = 0
  \item Number("abc") = NaN \textcolor{teal}{remember NaN is of type number!}
  \item one exception -> symbol, this is not a number\newline
    Error would be: Uncaught TypeError: Cannot convert a Symbol value to a number
  \vspace{-3mm}
\end{itemize}\\
\hline
String & 
\textcolor{teal}{Can be represented with either "" or ''}\newline
\textcolor{teal}{Escape character is \textbackslash}\newline
\minipg{
Typical Methods:\newline
\begin{itemize}
  \item length
  \item slice
  \item trim\char`\( \char`\)
  \item includes\char`\( \char`\)
  \item indexOf\char`\( \char`\)
\end{itemize}
}
{
  Special Operations:\newline
  \textcolor{red}{Number + String = String}\newline
  \textcolor{red}{String + Number = String}\newline
}\\
\hline
\end{tabular}
\end{table}
\pagebreak
\begin{table}[ht!]
\begin{tabular}{|m{0.2\linewidth}|m{0.755\linewidth}|}
\hline
null / undefined & 
\textcolor{teal}{Undefined means nothing, not yet defined}\newline
\textcolor{teal}{null is a value, the null value}\newline
\textcolor{red}{IMPORTANT: null == undefined = true !!}\newline
Example with null and undefined:\newline
\begin{lstlisting}
import * as userList
from './user.mjs';
console.log( userList.get(0) );
console.log( userList.update(0, {birthday: "19.05.1986"}) );
console.log( userList.update(0, {name : undefined, birthday: "19.05.1986"}) );
console.log( userList.update(0, {name : null, birthday: "19.05.1986"}) );
{ name: 'Michael', birthday: '19.05.1985' }
{ name: 'Michael', birthday: '19.05.1986' }
{ name: 'Michael', birthday: '19.05.1986' }
{ name: null, birthday: '19.05.1986' }
\end{lstlisting}
\\
\hline
Array & 
\begin{lstlisting}
const arr = [ 'a', 'b', 'c' ];
arr[0] = 'x';
arr.push("d");
console.log(arr); // [ 'x', 'b', 'c', 'd' ]
console.log(arr.length); // 4
\end{lstlisting}
\, \newline
This is the list for everything.\newline
\begin{itemize}
  \item \textcolor{teal}{no fix length}
  \item \textcolor{teal}{index starts at 0}
  \vspace{-3mm}
\end{itemize}
\, \newline
\, \newline
\textcolor{orange}{\textbf{To check if something is an array you can compare constructors!}}\newline
\begin{lstlisting}
function isArray(arr) {
  if(arr.constructor === Array) {
    return true;
  }
  return false;
}
\end{lstlisting}
\, \newline
\textcolor{teal}{arr.forEach}\newline
\begin{lstlisting}
let sum = 0;
arr.forEach(num => { sum += num });
\end{lstlisting}
\, \newline
\textcolor{teal}{arr.reduce}\newline
\begin{lstlisting}
const sum = arr.reduce((total, n) => total + n, 0);
\end{lstlisting}
\\
\hline
For Loops & 
\textcolor{teal}{classic:}\newline
\begin{lstlisting}
for(let i=0; i<arr.length; ++i) {
  console.log("for",arr[i]);
} // like regular loop just with let
\end{lstlisting}
\, \newline
\textcolor{teal}{For-In:}\newline
\begin{lstlisting}
for(const x in arr) {
  console.log("for in", x + ":" + arr[x]);
} // !! this returns the index string not the element !!
\end{lstlisting}
\, \newline
\textcolor{teal}{For-Of:}\newline
\begin{lstlisting}
for(const y of arr) {
  console.log("for of", y);
} // like for(auto e : arr)
\end{lstlisting}
\\
\hline
Object & 
\textcolor{teal}{An object is a collection of properties}\newline
These values are stored in a \textbf{key | value} -> \textbf{HashSet}.\newline
\begin{lstlisting}
cosnt person = {
  name: "spass"
  func: function() {
    return this.name;
  }
};
person.name = "Bob";
person.func();
\end{lstlisting}\\
\hline
Mutability of Objects & 
\textcolor{teal}{You can overwrite functions inside of these objects!}\newline
\begin{lstlisting}
person.func = function() {
  console.log("this does something else now !");
}
\end{lstlisting}\\
\hline
Functions &
\textcolor{teal}{As expected, functions are first class citizens, aka they can be variables!}\newline
\begin{lstlisting}
func = printSomething() {
  console.log("ping pang");
}
func();
\end{lstlisting}
\, \newline
Or You can use something like lambdas:\newline
\begin{lstlisting}
const func = (value) => {
  console.log(value);
}
func(5); // prints 5
\end{lstlisting}\\
\hline
\end{tabular}
\end{table}
\pagebreak
\begin{table}[ht!]
\begin{tabular}{|m{0.2\linewidth}|m{0.755\linewidth}|}
\hline
Parameters & 
\large\textcolor{red}{in javascript you can call functions with too many parameters, they will simply be stored in a buffer!!}\newline
\normalsize
\begin{lstlisting}
function foo(name, ...params){
console.log(1,name);
console.log(2,params.join(";"));
}
foo("Michael", "Gfeller", "OST", "IFS");
// prints "Michael" and "Gfeller;OST;IFS"
\end{lstlisting}\\
\hline
& \large \textcolor{teal}{Properties of functions}\newline
\normalsize
\begin{lstlisting}
const fn1 = function(){ return "Michael" };
console.log(fn1.name);
// ""

const fn2 = function name(){ return "Michael" };
console.log(fn2.name);
// "name"
console.log(fn2.length); // 0

const fn3 = function name(name){ return name };
console.log(fn3.length); // 1
\end{lstlisting}
\, \newline
\large \textcolor{red}{!!! Javascript doesn't have function overloading !!!}\newline
\normalsize The solution is to use if statements inside those functions with \textbf{typeof}\newline
\begin{lstlisting}
jQuery.fn.init = function( selector, context ) {
//...
if ( !selector ) {
return this;
}
// Handle HTML strings
if ( typeof selector === "string" ) {
//...
} else if ( selector.nodeType ) {
//...
} else if ( jQuery.isFunction( selector ) ) {
//...
}
//...
return jQuery.makeArray( selector, this );
};
\end{lstlisting}\\
\hline
\end{tabular}
\end{table}
\pagebreak
\begin{table}[ht!]
\section{DOM Document Object Model}
\begin{tabular}{|m{0.2\linewidth}|m{0.755\linewidth}|}
\hline
window & 
\textcolor{orange}{This simply provides global objects such as \textbf{console, document and HTMLDocument}}\newline
\textcolor{teal}{All global variables reside here.}\\
\hline
Selection &
\textcolor{orange}{There are different ways of selecting an element or node:}\newline
\begin{itemize}
  \item \textcolor{teal}{document.querySelector('name')} // select by CSS syntax -> also allows css selectors!
  \item \textcolor{teal}{document.querySelectorAll('.nav')} // select All by CSS syntax -> also allows css selectors!
  \item \textcolor{teal}{document.getElementById('list-container')} // select by id
  \item \textcolor{teal}{document.getElementByTagName('li')} // select by Tag name
  \item \textcolor{teal}{document.getElementsByClassName('nav-item')} // select by class name
\end{itemize}
\, \newline
\textcolor{orange}{Just use the querySelector -> it allows more ways to select it! \newline
However theoretically the html selectors are faster!}\\
\hline
DOM-Manipulation by create & 
\textcolor{orange}{We can also create new elements with js the same way we select elements}\newline
\begin{lstlisting}
const newEl = document.createElement('div');
newEl.appendChild(document.createTextNode('Hello'));
document.querySelector("#container").appendChild(newEl);
\end{lstlisting}
\, \newline
\pic{2022-10-18-11:26:10.png} \pic{2022-10-18-11:26:36.png}\newline
\textcolor{orange}{this is faster with small changes, DOM references stay the same, event handlers stay alive}\\
\hline
DOM-Manipulation by innerHTML & 
\begin{lstlisting}
const c = document.querySelector('#container');
c.innerHTML = '<div>Ping pang!</div>';
\end{lstlisting}
\, \newline
\textcolor{orange}{this is likely faster and more readable}\\
\hline 
DocumentFragment & 
\textcolor{orange}{This creates a temporary fragment that will be deleted when attaching this to a parent node.}\newline
\begin{lstlisting}
document.createDocumentFragment();
\end{lstlisting}
\, \newline
\pic{2022-10-18-11:31:01.png}\\
\hline
EventListener & 
\textcolor{orange}{An event listener is used to bind a function to something like a button\newline
For example a darkmode button is mapped to the useDarkMode function}\newline
\begin{lstlisting}
document.querySelector('button')
.addEventListener('click' // action to map , useDarkMode // function to map,[options] // once, capture etc)
\end{lstlisting}
\\
\hline
remove Eventlistener \newline and dispatchEvent & 
You can of course also remove an eventlistener, or execute an event per js call:\newline
\begin{lstlisting}
document.querySelector('button').removeEventListener('click', useDarkMode);
// remove eventlistener

document.querySelector('button').dispatchEvent('click');
// dispatch the click event
\end{lstlisting}\\
\hline
NodeStructure & 
\textcolor{orange}{HTML is built like a tree with different nodes and leafs.}\newline
\textcolor{teal}{Important is that there are different types of nodes that html uses}\newline
\begin{itemize}
  \item \textcolor{orange}{ElementNode} head,body,li,title etc
  \item \textcolor{orange}{AttributeNode} href, charset, src etc
  \item \textcolor{orange}{TextNode} some text to display
  \item \textcolor{orange}{Comments} <!-- kekw -->
  \vspace{-3mm}
\end{itemize}
\, \newline
\pic{2022-10-18-10:35:52.png}
\\
\hline
\end{tabular}
\end{table}
\pagebreak
\begin{table}[ht!]
\begin{tabular}{|m{0.2\linewidth}|m{0.755\linewidth}|}
\hline
\textbf{async and defer} & 
\textcolor{orange}{\textbf{async:} Async means that the script will be loaded and executed immediately.\newline
This means that there \textbf{is no guarantee for order} and the script will be \textbf{executed before the document has loaded}.}\newline
\textcolor{blue}{\textbf{defer:} This is the opposite to async, it will \textbf{gurarantee order}, \newline
and the document will \textbf{be loaded before the script executes}.}\newline
\pic{2022-10-18-10:45:27.png}\includegraphics[scale=0.4]{2022-10-18-10:46:00.png}\newline
\textcolor{orange}{\textbf{If you want to check whether or not the document is ready, \newline
then you can do this with: }}\textcolor{red}{\textbf{\emph{document.readyState}}}
\\
\hline
TextContent and innerText & 
\textcolor{orange}{HTMLElement.innerText provides the rendered text with css applied}\newline
\textcolor{blue}{HTMLElement.textContent provides the complete raw text with no css applied}\\
\hline 
className and classList & 
\begin{lstlisting}
<script>
console.log(
document.querySelector("#el").className);
// box alert important

console.log(document.querySelector("#el")
.classList);
// DOMTokenList(3) ["box", "alert", "important"]
</script>
\end{lstlisting}\\
\hline
\end{tabular}
\subsection{Event Handling}
\begin{tabular}{|m{0.2\linewidth}|m{0.755\linewidth}|}
\hline
EventListener with options & 
\begin{lstlisting}
target.addEventListener(type, listener[, options]);
\end{lstlisting}
\, \newline
options:\newline
\begin{itemize}
  \item capture \newline
    Whether or not the element in question should catch the event during capture phase
  \item once \newline
    Whether or not the eventhandler should function \textbf{only once}
  \item passive \newline
    Specifies that the function in question \textbf{will not call PreventDefault()}
  \item signal \newline 
    Specifies that the eventlistener will be removed if the \textbf{AbortSignals associated abort() method is called}
  \vspace{-3mm}
\end{itemize}\\
\hline
Event Listener vs inline & 
\begin{lstlisting}
<HTMLElement>.addEventListener()
document.querySelector("#1").addEventListener("click", () => alert('1'));
// multiple listeners possible!

<HTMLElement>.on... = eventHandler
document.querySelector("#2").onclick = () => alert('2');
// only 1 listener possible
// will overwrite previous onclick assignments

<button onclick="alert('3')">3</button>
// provides the least amount of flexibility, not recommended
\end{lstlisting}\\
\hline
Event Phases & \minipg{
\textcolor{orange}{Events go through 3 phases:}\newline
\begin{enumerate}
  \item \textcolor{teal}{Capture-Phase}\newline
    Event travels from root to leaf\newline
    Every Element can react here
  \item \textcolor{teal}{Target-Phase}\newline
    Event will be destroyed on target
  \item \textcolor{teal}{Bubble-Phase}\newline/
    Event travels from leaf to root\newline
    Each element can react
\end{enumerate}
\, \newline
\textcolor{orange}{Event bubbling and capture is used to dynamically change lists, etc.}
}
{\pic{2022-10-18-12:15:49.png}}[0.3,0.4]\\ 
\hline
\end{tabular}
\end{table}
\pagebreak
\begin{table}[!ht]
\subsection{Node}
\begin{tabular}{|m{0.2\linewidth}|m{0.755\linewidth}|}
\hline
Important Properties and functions & 
\textcolor{orange}{\textbf{Properties}}
\begin{itemize}
  \item \textcolor{teal}{children} only nodes of type HTMLelement 
  \item \textcolor{teal}{childNodes} 
  \item \textcolor{teal}{firstChild} first node child
  \item \textcolor{teal}{firstElementChild} first element child
  \item \textcolor{teal}{nextSibling} next node sibling
  \item \textcolor{teal}{nextElementSibling} next element 
  \item \textcolor{teal}{parentElement} 
\end{itemize}
\, \newline
\textcolor{orange}{Functions}\newline
\begin{itemize}
  \item \textcolor{teal}{appendChild()}
  \item \textcolor{teal}{removeChild()}
  \vspace{-3mm}
\end{itemize}\\
\hline
\end{tabular}
\subsection{Element}
\begin{tabular}{|m{0.2\linewidth}|m{0.755\linewidth}|}
\hline
Important Properties and functions & 
\textcolor{orange}{\textbf{Properties}}
\begin{itemize}
  \item \textcolor{teal}{id} 
  \item \textcolor{teal}{className} 
  \item \textcolor{teal}{classList} 
  \item \textcolor{teal}{innerHTMl}
\end{itemize}
\, \newline
\textcolor{orange}{Functions}\newline
\begin{itemize}
  \item \textcolor{teal}{getAttribute()}
  \item \textcolor{teal}{setAttribute()}
  \item \textcolor{teal}{toggleAttribute()}
  \item \textcolor{teal}{closest()}
  \vspace{-3mm}
\end{itemize}\\
\hline
Element ParentNode & 
\textcolor{orange}{Element also implements Parentnode:}\newline
Properties:\newline
\begin{itemize}
  \item \textcolor{teal}{children}
  \item \textcolor{teal}{firstElementChild}
  \item \textcolor{teal}{lastElementChild}
\end{itemize}
\, \newline
Functions:\newline
\begin{itemize}
  \item \textcolor{teal}{append()} 
  \item \textcolor{teal}{remove()}
  \vspace{-3mm}
\end{itemize}\\
\hline
\end{tabular}
\subsection{HTMLElement}
\begin{tabular}{|m{0.2\linewidth}|m{0.755\linewidth}|}
\hline
Important Properties and functions & 
\textcolor{orange}{\textbf{Properties}}
\begin{itemize}
  \item \textcolor{teal}{dataset} 
  \item \textcolor{teal}{style} 
  \item \textcolor{teal}{hidden}
\end{itemize}
\, \newline
\textcolor{orange}{Functions}\newline
\begin{itemize}
  \item \textcolor{teal}{createCaption()}
  \item \textcolor{teal}{createTFoot()}
  \item \textcolor{teal}{createTHead()}
  \vspace{-3mm}
\end{itemize}\\
\hline
\end{tabular}
\subsection{Event-Object}
\begin{tabular}{|m{0.2\linewidth}|m{0.755\linewidth}|}
\hline
Important Properties and functions & 
\textcolor{orange}{\textbf{Properties}}
\begin{itemize}
  \item \textcolor{teal}{target} // element of event origin 
  \item \textcolor{teal}{currentTarget} // element that has listener for this event
\end{itemize}
\, \newline
\textcolor{orange}{Functions}\newline
\begin{itemize}
  \item \textcolor{teal}{preventDefault()} // prevents default actions like automatic form submit
  \item \textcolor{teal}{stopPropagation()} // stops capturing and bubbling
\end{itemize}
\, \newline
\textcolor{orange}{Specific Event types}\newline
\begin{itemize}
  \item \textcolor{teal}{MouseEvent}
  \item \textcolor{teal}{WheelEvent}
  \item \textcolor{teal}{InputEvent}
  \item \textcolor{teal}{KeyboardEvent}
  \vspace{-3mm}
\end{itemize}\\
\hline
Keyboard Event & 
\begin{lstlisting}
<body>
<input>
<script>
document.querySelector("input").addEventListener("keydown", (event) => {
console.log(event.key);
})
</script>
\end{lstlisting}
\minipg{
\begin{itemize}
  \item change: what changed?
  \item keydown: which key has been pressed?
  \item ctrlKey: was the control pressed during keydown?
  \vspace{-3mm}
\end{itemize}
}
{\includegraphics[scale=0.5]{2022-10-18-12:33:39.png}}[0.35,0.4]\\
\hline
\end{tabular}
\end{table}
\pagebreak
\begin{table}[ht!]
\section{Scopes}
\begin{tabular}{|m{0.2\linewidth}|m{0.755\linewidth}|}
\hline
Var vs let & 
\textcolor{orange}{Usually we declare variables using the let keyword, however, the old var keyword still has one technicality that the let keyword doesn't have. Namely the var keyword will always be available for the parent scope:}\newline
\begin{lstlisting}
{ // empty explicit scope to show difference
  function some_func() {
    let num1 = 5;
    var num2 = 6;
    console.log(num1); // ok -> 5
    console.log(num2); // ok -> 6
  }
    console.log(num1); // -> undefined!
    console.log(num1); // ok -> 6
}
\end{lstlisting}\\
\hline
Scope per file in NodeJs &
\textcolor{orange}{NodeJs creates an additional scope per file, this means that you essentially can only explicitly declare a truly global variable with var.}\newline
\textcolor{red}{However, the browser does \textbf{NOT} have this feature!}\\
\hline
Closure & 
\textcolor{orange}{This is just the concept of boxing a something in something and then accessing the scope from the inner "something". Usually done with functions:}\newline
\begin{lstlisting}
function something1() {
  let num = 5;
  function something2() {
    console.log(num); // obviously works
  }
}
\end{lstlisting}\\
\hline
global Scope & 
\textcolor{orange}{If you access a global variable, it is probably better to use the \textbf{global.variable} naming scheme. The reason for this is obviously the fact that you don't use the variable of another scope by accident.}\\
\hline
This in scopes & 
\textcolor{orange}{if we do not have an object, then the \textbf{the global scope will be this!}\newline
This means that you can do things like this:}\newline
\begin{lstlisting}
function printName() {
  console.log(this.name);
}

const name = "ping"

const logEntry = {
  name: "pang"
};

logEntry.printName = printName;
logEntry.printName(); // will print "pang"
printName(); // will print "ping" -> empty object will always invoke the global object!!
\end{lstlisting}\\
\hline
Functions with new (old js) & 
\textcolor{orange}{Before proper OOP was introduced, this was the way you wrote OOP in js:}\newline
\begin{lstlisting}
// given the code from above, we create another printName instance with new
// this will create an empty object that will be the "instance" to call the method on
new printName();

// This is what happens under the hood when you call new printName()!
// function newPrintName() {
//   var self = {}
//   console.log(self.name);
//   return self;
// }
\end{lstlisting}\\
\hline
This as a parameter & 
\textcolor{orange}{You can also pass the this keyword as a parameter!}\newline
\begin{lstlisting}
logEntry.printName({name: "pingpang"});
// this will print "pingpang" instead of the logEntry name "pang" !!
\end{lstlisting}\\
\hline
Bind & 
\textcolor{orange}{You can also "bind" a certain "this" to one function in particular, this means that all other function other than the one you bound, will be used with the regular "this"!!}\newline
\begin{lstlisting}
const bindPrintName = logEntry.printName({ name: "burrmiu" });
bindPrintName(); // this will ALWAYS print "burrmiu"
// at least until you override it!
bindPrintName({name: "pangPing!"}); // !!!! this will still print "burrmiu" !!!!
\end{lstlisting}\\
\hline
\end{tabular}
\section{strict}
\begin{tabular}{|m{0.2\linewidth}|m{0.755\linewidth}|}
\hline
Basic & 
\textcolor{orange}{This disables certain JS features that might be unwanted in this particular usecase. \newline
It can be enabled per Scope or per file}\newline
\textcolor{purple}{Usually this simply makes JS throw an exception when you would expect it from other functions,for example when using a method with a "this" but used on nothing -> exception instead of using global scope.}\\
\hline
Usage & 
\begin{lstlisting}
function someFunc() {
  'use strict'; // this enables the strict mode for this particular scope or file
  // for compatability, should a browser use old js without this feature, then the strict will simply be ignored
}
\end{lstlisting}\\
\hline
General & 
\textcolor{orange}{All new features in JS are in strict mode, this essentially forces us to use it, and it is generally good practice to simply use the strict mode!}\\
\hline
\end{tabular}
\end{table}
\pagebreak
\begin{table}[ht!]
\section{Lambda}
\begin{tabular}{|m{0.2\linewidth}|m{0.755\linewidth}|}
\hline
Notation & 
\begin{lstlisting}
let x = (parameter) => {
  // Do something
  console.log(parameter);
};
x(5); // will print 5!
\end{lstlisting}\\
\hline
This in Lambda &
\textcolor{orange}{In Lambdas the this is always based on the parent scope.\newline
For example within a function or a class the this will be based on that function or class:}\newline
\begin{lstlisting}
function Point(x, y){
this.x = x;
this.y = y;
this.area = () => this.x + this.y;
}

var areaFn = new Point(10,50).area;
console.log(areaFn());

// explicit version!
// function Point(x, y) {
//   var _this = this; // here the _this is mapped to this!
//   this.x = x;
//   this.y = y;
//   this.area = function () {
//     return _this.x + _this.y;
//   };
// }
\end{lstlisting}\\
\hline
\end{tabular}
\section{Objects and Classes}
\begin{tabular}{|m{0.2\linewidth}|m{0.755\linewidth}|}
\hline
Objects are Hashtables! & 
\textcolor{orange}{Every object is a hashtable, this means that we have different means to access it's values. \newline
The same thing therefore applies to arrays as well as long as there is a name for it:}\newline
\begin{lstlisting}
const items = {name: "pingpang"};

console.log(items.name);
console.log(items["name"]);
// these are both the same!
\end{lstlisting}
\, \newline
\textcolor{purple}{Since js does not have static types, you can also add something like name: "ping" to an array and access it the same way you access any other object, however this is \textbf{not recommended!}}\\
\hline
Classes &
\vspace{2mm}
\begin{itemize}
\item \textcolor{purple}{\#: This is the prefix for private variables}
\item \textcolor{purple}{super: Superclass call}
\item \textcolor{purple}{instanceof: this checks whether or not,\newline
  an instance is an object of class xyz\newline
Note, subclasses are also instanceof parentclass!}
\vspace{-3mm}
\end{itemize}
\begin{lstlisting}
class Clock {
  #timer // the # makes it private!
  currentTime
  constructor(param) {
    super();
  }
  start() {
    this.#timer = setTimeout(() => { 
      this.currentTime = new Date();
    }, 1000);
  }
  get time() {
    return this.currentTime
  }
  set time(newTime) {
    this.currentTime = newTime
  }
}

const clock = new Clock(); // instance
\end{lstlisting}\\
\hline
Classes before JS6 & 
\textcolor{orange}{The prototype keyword was used to use OOP before JS6, this is still used under the hood, but do not use this anymore unless you have performance issues.}\\
\hline
This Context with Lambda Callbacks &
\textcolor{red}{This code below does not work, the reason for this is that when we pass person.wackUP, we lose the this context, as we only pass the function itself!}\newline
\begin{lstlisting}
class Person {
  constructor(name) {
    this.name = name;
  }

  wackUp() {
    console.log(`${this.name} is awake!`)
  }
}
class Alarm {
  registerAlarm(callback) {
    setTimeout(() => {
      callback()
    }, 1000);
  }
}
const person = new Person("Michael");
const alarm = new Alarm();
alarm.registerAlarm(person.wackUp);
\end{lstlisting}
\\
\hline
\end{tabular}
\end{table}
\pagebreak
\begin{table}[ht!]
\begin{tabular}{|m{0.2\linewidth}|m{0.755\linewidth}|}
\hline
&
\textcolor{red}{In order to preserve the context, we can pass the function as a lambda:}\newline
\begin{lstlisting}
alarm.registerAlarm(() => person.wackUP() );
\end{lstlisting} 
\, \newline
\textcolor{red}{The same can be done in the class itself!!}\newline
\begin{lstlisting}
wackUp = () => {
  console.log(`${this.name} is awake!`)
}
\end{lstlisting}\\
\hline
\end{tabular}
\section{Modules}
\begin{tabular}{|m{0.2\linewidth}|m{0.755\linewidth}|}
\hline
UseCase & 
\textcolor{orange}{The use case is as expected the prevention of namespace issues.\newline
For example every script in the browser is in the global scope, this means that if the index.js and the home.js both have a function called hello(), then they will overwrite each others definition!!}\newline
\textcolor{purple}{The other usecase is \textbf{dependency solving} and \textbf{code departmentalization}.}\newline
\textcolor{teal}{In Node.js the file need to end with .mjs in order to use modules, in the browser the ending .js is fine.}\newline
\textcolor{teal}{Modules are \textbf{ALWAYS STRICT!}}\\
\hline
Definition & 
\textcolor{purple}{\textbf{Modules simply export or import values and functionality to or from other modules!}}\\
\hline
Module Usage & 
\textcolor{red}{\textbf{In order to use a module, the file itself needs to be a module!}}\newline
\begin{lstlisting}
<script src="index.js" type="module"></script>
\end{lstlisting}
\, \newline
\textcolor{orange}{The usage is done with the import keyword:}\newline
\begin{lstlisting}
import {register as alarmClock} from './libs/alarm-clock.js'
import {register as newsFeed} from './libs/news-feed.js'
import * from './pingpang.js' // import every export from this file 
import defaultExport from './something.js' // default export explained below
\end{lstlisting}\\
\hline
Export& 
\textcolor{purple}{There are 2 types of exports, \textbf{named} and \textbf{unnamed}.}\newline
\, \newline
\textcolor{orange}{\textbf{Named Export}: In order to export a function or value, you need to use the export keyword before that function/value:}\newline
\begin{lstlisting}
export someFunc() {
  console.log("pingpang");
}
\end{lstlisting}
\, \newline
\textcolor{orange}{\textbf{defaultExport}: Here we define an export at the end of the file:}\newline
\begin{lstlisting}
export default {name: "pangping"};

// in the import file
console.log(defaultExport.name);
\end{lstlisting}
\\
\hline
\end{tabular}
\section{Advanced JS Features}
\begin{tabular}{|m{0.2\linewidth}|m{0.755\linewidth}|}
\hline
Spread & 
\textcolor{orange}{Spread is a feature that allows fast and easy concatenation of arrays just like in haskell:}\newline
\begin{lstlisting}
const listA = [1,2,3,4,5];
const listB = [10,11,12,13,14];
const listC = [...listA, ...listB];

// this can also be done with other objects

const objA = {name: "pingpang"};
const objB = {name: "pangping"};
const objC = {...objA,...objB}; // this copies the values of objA and objB into objC
const objD = {objA,objB}; // this however copies the full object into it

console.log(objC); // {name: "pingpang", name: "pangping"}
console.log(objD); // {{name: "pingpang"}, {name: "pangping"}}
// note the extra curly braces on the objD -> deep copy!
\end{lstlisting}\\
\hline
Spread Operator as parameter & 
\textcolor{orange}{You can also enter an array as a parameter:}\newline
\begin{lstlisting}
function something(a,b,c) {
  return a + b + c;
}
const arr = [1,2,3,4,5,6,7,8,9];
something(...arr); // this works -> a=1,b=2,c=3, rest ignored
\end{lstlisting}\\
\hline
Destructuring & 
\textcolor{orange}{You can create variables directly from an array or object:}\newline
\begin{lstlisting}
const [a, b] = [10,20]; // a = 10, b = 20 
const {c, d} = {name: "pang", price: 50};

function something({message}) {
  console.log(message);
}
something({message:"pangping"}) // prints "pangping"!

function printWithDefaults({message = "message"} = {}) { // object destructuring
console.log(message);
}
printWithDefaults({code: "error_1"}) // error_1
printWithDefaults() // message
\end{lstlisting}\\
\hline
Nullish Operator & 
\textcolor{orange}{In case you want to check if for example a number exists, even a 0, you can use the nullish operator.\newline
This will simply use the value in case it exists, or the default value that you have specified if it is undefined or null.}\newline
\begin{lstlisting}
console.log(0 ?? 42); // will print 0 
console.log(undefined ?? 42); // will print 42
\end{lstlisting}\\
\hline
\end{tabular}
\end{table}
\pagebreak
\begin{table}[ht!]
\begin{tabular}{|m{0.2\linewidth}|m{0.755\linewidth}|}
\hline
Optional chaining & 
\textcolor{orange}{In JS you might always encounter values that are undefined, this often leads to issues where you call a function with undefined values, or even worse the entire function is undefined.\newline
We can counteract this with the Optional Operator:}\newline
\begin{lstlisting}
const item = {
  name: "pangping",
  price: 10, 
  fun: {
    word: "no"
  }
};
console.log(item.fun?.word); // this will print "no" IF the fun object exists!
// otherwise we simply do not execute this code, no exception will be thrown.
\end{lstlisting}\\
\hline
\end{tabular}
\section{Clean Code}
\begin{tabular}{|m{0.2\linewidth}|m{0.755\linewidth}|}
\hline
Naming & 
\textcolor{orange}{Names of variables, function etc should be \textbf{short, intuitive and explicit}.}\newline
\textcolor{teal}{Also the names should be \textbf{spellable} and consisting of \textbf{kown words}, something that you can't say is useless in coop programming.}\newline
\textcolor{purple}{Avoid repeating yourself, if the app is about webtrc, you don't need to specify this every time!}\\
\hline
Naming schemes & 
\textcolor{orange}{There are different schemes for different types of variables that you use.}\newline
\begin{itemize}
\item \textcolor{purple}{regular variables and functions >> camelCase}
\item \textcolor{purple}{constants >> ALL\_CAPS}
\item \textcolor{purple}{boolean values >> isValue}
\item \textcolor{purple}{arrays >> pluralNaming}
\vspace{-2mm}
\end{itemize}\\ 
\hline
Functions & 
\textcolor{orange}{Function should also be \textbf{short, fitting name for functionality, pos-x compliant} -> only 1 functionality and \textbf{as few parameters as possible}}\\
\hline
Code Smells & 
\begin{tabular}{m{0.3\linewidth}m{0.45\linewidth}}
Names\newline
\begin{itemize}
\item \textcolor{orange}{useless Names}
\item \textcolor{orange}{too short names}
\item \textcolor{orange}{too abstract names}
\end{itemize}
& 
Functions\newline
\begin{itemize}
\item \textcolor{orange}{too many parameters}
\item \textcolor{orange}{too much dead code -> not reached code}
\item \textcolor{orange}{Function too long}
\item \textcolor{orange}{boolean flags}
\end{itemize} \\
Comments\newline
\begin{itemize}
\item \textcolor{orange}{outdated comments}
\item \textcolor{orange}{superfluous comments}
\item \textcolor{orange}{commented code}
\end{itemize} 
& 
More \newline
\begin{itemize}
\item \textcolor{orange}{Code duplicates}
\item \textcolor{orange}{Inconsistent formatting}
\item \textcolor{orange}{Inconsistent naming across functions and names}
\end{itemize}\\
\end{tabular}
\\
\hline

\hline

\hline

\hline

\hline
\end{tabular}
\end{table}
\pagebreak
\begin{table}[ht!]
\section{PolyFill and BabelJS}
\begin{tabular}{|m{0.2\linewidth}|m{0.755\linewidth}|}
\hline
BabelJS & 
\textcolor{orange}{BabelJS converts new and modern JS code into old style JS, this is used for compatibility with older browsers that do not support the new features.}\\
\hline
PolyFill &
\textcolor{orange}{BabelJS has certain limitations that PolyFill tries to essentially "fill in". This further increases the compatibility with older browsers, however, this still is limited to a certain extend.\newline
For example due to the translating, the performance will be much worse, and considering we are already using an older browser, this can quickly get out of hand.}\\
\hline
Model View Controller (MVC) & 
\textcolor{orange}{This is essentially an abstract system to create web-applications, it abstracts 3 different things: the \textbf{view -> Display of data}, the \textbf{model -> app state}, and the \textbf{Controller -> reaction to input}}\newline
\includegraphics[scale=0.4]{2022-11-08-11:39:15.png}\newline
\textcolor{teal}{You can create this for every project that you would create with the tri-force html/css/js}\\
\hline
Example for MVC & 
\vspace{2mm}
\includegraphics[scale=0.4]{2022-11-08-11:42:52.png}\newline
\includegraphics[scale=0.4]{2022-11-08-11:42:58.png}
\includegraphics[scale=0.4]{2022-11-08-11:43:08.png}\\
\hline
\end{tabular}
\section{npm}
\begin{tabular}{|m{0.2\linewidth}|m{0.755\linewidth}|}
\hline
Basics Commands & 
\vspace{2mm}
\begin{itemize}
\item \textcolor{purple}{npm init} -> create a new project
\item \textcolor{purple}{npm install} -> install all packages needed for this project
\item \textcolor{purple}{npm run} -> run the project
\vspace{-3mm}
\end{itemize}\\ 
\hline
Packages in npm & 
Each node package has its own \textbf{package.json}.\newline
This specifies \textbf{the name, the type, the version, main file, description and dependencies}:\newline
Here an example package:\newline
\begin{lstlisting}
{
  "name": "my-package",
  "type": "module",
  "version": "1.0.0",
  "main": "index.js",
  "description": "My first node package",
  "dependencies": {
    "ws": "^7.3.1"
  }
  "devDependencies": {
    "stylelint": "^13.13.1"
  }
}
\end{lstlisting}\\
\hline
Creating a small Server & 
\begin{lstlisting}
// imports and config
import http from 'http';
const hostname = '127.0.0.1'
const port = 3000

// handler for requests
const server = http.createServer((req, res) => {
  res.statusCode = 200
  res.setHeader('Content-Type', 'text/plain')
  res.end('Hello World\n')
})

// start the server
server.listen(port, hostname, () => {
  console.log(
  `Server running at http://${hostname}:${port}/`)
})
\end{lstlisting}\\
\hline
\end{tabular}
\end{table}
\pagebreak
\begin{table}[ht!]
\section{AJAX Asynchronous JavaScript and XML}
\begin{tabular}{|m{0.2\linewidth}|m{0.755\linewidth}|}
\hline
Use Case & 
The problem with js is that it is singlethreaded, this means that only one thing can be done at a time.\newline
This makes things like animations problematic, as they can be blocked by requests, clicks etc.\newline
To solve this the async engine handles a "queue", which means that a request might \textbf{not be executed immediately, but whenever there is a "free space for it"}\newline
Other Effects that AJAX made possible:\newline
\begin{itemize}
\item \textcolor{purple}{Single Page Applications}
\item \textcolor{purple}{Animations and other changes without refreshing}
\item \textcolor{purple}{Data via "best effort", UI via "guaranteed"}
\vspace{-3mm}
\end{itemize}\\ 
\hline
Promise & 
This states that \textbf{at some point you will receive an answer}, however, \textbf{there is no guarantee for the value}, it can be anything or even null/undefined.\newline
This is mostly used to get a value that you can't be sure to get, consider it similar to channels in async rust, just that this is within the same thread!\newline
\textcolor{OliveGreen}{A promise has 3 states, \textbf{Fulfilled, Pending and Rejected}, fulfilled means the action was successfull and a return value is available, pending means operation still active and rejected means the action has failed}\newline
\begin{lstlisting}
async function fetchFromURL() {
  return new Promise ((resolve, reject) => {
  const response = fetch('https://stone.sifs0005.infs.ch/ranking')
    .then(res => res.json())
    .then(json => resolve(json))
  }).catch(
    err => reject(err)
  );
}
\end{lstlisting}\\
\hline
Constructor of Promise & 
\begin{lstlisting}
new Promise((resolve, reject) =>
// call async func with arguments
>>>async fn<<< (..., (...callbackArgs) => {
  if (error) {
    reject(error);
  } else {
    resolve(value);
  }
})
\end{lstlisting}
\, \newline
As you can see a promise has 2 functions as parameters that will handle the 2 possible end states reject -> rejected and resolve -> fulfilled.\\
\hline
Using promises & 
You can either use promises with the \textbf{.then()} function, or with the \textbf{await} keyword.\newline
\begin{lstlisting}
// with .then()
function onClick() {
  feetch(url).then(
    (response) => {
      console.log(response);
    }
  );
}

// with await
async function fetchFromURLAwait() {
  const response = await fetch('https://stone.sifs0005.infs.ch/ranking');
  return response.json();
}
\end{lstlisting}\\
\hline
Try catch with async await & 
Since the promise can fail, we can use a try catch block to do error handling:\newline 
\begin{lstlisting}
async function getGreeting() {
  try {
    return await greeting;
  }
  catch (e) {
    console.error(e);
  }
}
\end{lstlisting}\\
\hline
\end{tabular}
\end{table}
\pagebreak
\begin{table}[ht!]
\section{REST}
\begin{tabular}{|m{0.2\linewidth}|m{0.755\linewidth}|}
\hline
Representational State Transfer (REST) & 
This is a style of transerring resources without side effects, this means that \textbf{we only transer data, there will be no function calls!}.\newline
\begin{itemize}
\item \textcolor{purple}{Client-Server based}
\item \textcolor{purple}{Stateless communication}
\item \textcolor{purple}{Cacheable}
\item \textcolor{purple}{Layered System} Load balancers, proxies and firewalls
\item \textcolor{purple}{Unified Interface}
\vspace{-3mm}
\end{itemize} \\
\hline
Richardsons Maturity Model & 
\vspace{2mm}
\includegraphics[scale=0.4]{2022-11-29-11:08:34.png}\\
\hline
Level 0: The Swamp of POX & 
Here we \textbf{send each request individually}, we send plain \textbf{xml} messages.\newline
Each request is sent to the \textbf{same endpoint}. 
\includegraphics[scale=0.35]{2022-11-29-11:14:31.png}\\
\\
\hline
Level 1: Ressources & 
With this level we can receive data from different endpoints, this essentially allows \textbf{divide and conquer} -> "multithreading".\newline 
\includegraphics[scale=0.35]{2022-11-29-11:14:57.png}\\
\hline
Level 2 HTTP-Verbs & 
Now we can use different html methods like \textbf{GET or POST}, this distinguishes the idea of pushing or getting data, making it easier to handle data apropriately. \newline
\includegraphics[scale=0.35]{2022-11-29-11:17:08.png}\\
\hline
Level 3: Hypermedia Controls & 
We now can include \textbf{states of the application}, this allows us to specify the exact usecase even further, making it easier to handle the data accordingly:\newline
Level 3 also gives us the ability to \textbf{specify possible actions} with this data, eg. book an appointment via an \textbf{URI}.
\includegraphics[scale=0.35]{2022-11-29-11:19:32.png} \newline
\textcolor{red}{Note the link rel and uri, which specify possible actions to take!}\\
\hline
\end{tabular}
\end{table}
\pagebreak
\begin{table}[ht!]
\begin{tabular}{|m{0.2\linewidth}|m{0.755\linewidth}|}
\hline
REST vs RPC (Remote Procedure Call) & 
REST is much simpler, this allows a much cleaner implementation of function on your side, and also gives you much more flexibility as to what and how you implement things. \newline
The downside is that at some point RPC will be easier when you get to a certain dependency on each other, in other words, loosely connected applications benefit from the REST concept, while strongly connected ones benefit from RPC.\newline 
\textcolor{orange}{In general REST is better for data requests, as it is based on ressources instead of calls}\\
\hline
Ressource Oriented Architectures & 
\vspace{2mm}
\begin{itemize}
\item \textcolor{purple}{Ressource}\newline
  Everything that is important enough to get its own id
\item \textcolor{purple}{Name}\newline
  unique ID of a ressource
\item \textcolor{purple}{Representation}\newline
  Can be done in multiple ways, html, xml, whatever
\item \textcolor{purple}{Links}\newline
  Hyperlinks connect ressources
\item \textcolor{purple}{Interface}\newline
  Uniform Interface\newline
  Standard HTML methods
\item \textcolor{purple}{Stateless Communication}
\vspace{-3mm}
\end{itemize} \\
\hline
HTTPS Request & 
\vspace{2mm}
\includegraphics[scale=0.4]{2022-11-29-11:35:27.png}\\
\hline
Query filter & 
You can sort your GET queries like this: \newline
\begin{lstlisting}
orders?state=active&seller_id=1234
\end{lstlisting}\\
\hline
Important HTTP Methods & 
\vspace{2mm}
\begin{itemize}
\item \textcolor{purple}{GET}\newline
  Request ressources
\item \textcolor{purple}{POST}\newline
  Create ressources
\item \textcolor{purple}{PUT}\newline
  Update or Create ressources
\item \textcolor{purple}{DELETE}\newline
  Delete ressources
\item \textcolor{purple}{PATCH}\newline
  Partial updates
\vspace{-3mm}
\end{itemize} \\
\hline
Best practices & 
\vspace{2mm}
\begin{itemize}
\item \textcolor{black}{Use JSON as it is native to JS}
\item \textcolor{black}{Use correct HTML methods}
\item \textcolor{black}{Use html status codes}
\item \textcolor{black}{API is only as good as its documentation}
\item \textcolor{black}{Limit data sent and received}
\vspace{-3mm}
\end{itemize}\\ 
\hline
Nielsen & 
This is a system of things to comply with:\newline
\begin{itemize}
\item \textcolor{purple}{Visibility of System State}\newline
  For example, display the steps that a program does when the user waits, instead of a generic message saying "please wait for shit to finish"
\item \textcolor{purple}{Connection between real world and system}
\item \textcolor{purple}{Freedom and Control for the user}
\item \textcolor{purple}{Consistency and Standards}
\item \textcolor{purple}{Avoiding Errors}
\item \textcolor{purple}{Remember states instead of asking about them}
\item \textcolor{purple}{Flexibility and efficient Usage}
\item \textcolor{purple}{Asthetics and minimalistic Design}
\item \textcolor{purple}{Support of Detection and Resolution of Problems}
\item \textcolor{purple}{Support and Documentation}
\vspace{-3mm}
\end{itemize} \\
\hline
Suspense State & 
In order to not annoy the user with a "suppossedly broken system", when in actuality the system just needs some time to make the request on the server, \textbf{consider imeplementing a \textcolor{red}{progress bar} inn order to communicate to the user, that we are currently working on the request.}\newline
\includegraphics[scale=0.4]{2022-12-06-11:30:22.png}\\
\hline
\end{tabular}
\end{table}
\pagebreak
\begin{table}[ht!]
\section{Usability}
\begin{tabular}{|m{0.2\linewidth}|m{0.755\linewidth}|}
\hline
Peripheral vs Central Vision & 
The idea is that changes outside of the central vision are hard to notice, meaning that it is often not recommended to show errors and changes outside of that vision, however, sometimes too much things popping up in the central vision can be irritating when doing some work, see "fabio rages yet again at steams fucking popup for an update"\newline
\includegraphics[scale=0.3]{2022-12-13-10:37:39.png}\\
\hline
UI element Intuitiveness & 
In general UI elements should work as expected, text fields should take inputs, buttons should be clickable, radio buttons should only allow one selection, etc.\newline
\textcolor{purple}{The easiest way to guarantee this is to make a connection to the real world, does it work the way everything else does?}\\
\hline
Consistency & 
Do not make sudden changes with your design language or your UI elements, it should be consistent all across your website, buttons should always work the same way, radio buttons should always work the same way, etc.\\
\hline
Rules of design & 
\vspace{2mm}
\begin{itemize}
\item \textcolor{purple}{Rule of Similarity}\newline
  \begin{itemize}
  \item \textcolor{black}{Similar things need to close together}
  \item \textcolor{black}{Based on colors, sizes, shapes, movements}
  \item \textcolor{black}{Generate contrasts to other elements}
  \item \textcolor{black}{Take level of similarity into account -> amount of similarities}
  \end{itemize} 
\item \textcolor{purple}{Rule of Distance}\newline
  \begin{itemize}
  \item \textcolor{black}{Things that are close together will be considered a group by the human eye}
  \item \textcolor{black}{Therefore neighbours should always be in a group}
  \item \textcolor{black}{Make clear distinctions from one group to the other}
  \end{itemize} 
\item \textcolor{purple}{Rule of Conciceness}\newline
  You should not provide unnecessary information and UI elements -> material design
\item \textcolor{purple}{Rule of Simplicity}\newline
  Less is often more, again -> Material Design
\item \textcolor{purple}{Rule of equal direction}\newline
  Items that are grouped should be facing the same direction
\item \textcolor{purple}{Rule of Closed system}\newline
Your design should be considered a closed system, meaning that it does not use the entirety of the screen
\vspace{-3mm}
\end{itemize}
\, \newline
\includegraphics[scale=0.4]{2022-12-13-11:37:23.png}\\
\hline
Colors & 
\vspace{2mm}
\begin{itemize}
\item \textcolor{purple}{Use less than 6 colors}
\item \textcolor{purple}{Use colors in the general application}\newline
  green for go, red for stop
\item \textcolor{purple}{Make sure contrasts make sense}
\item \textcolor{purple}{use color schemes}
\vspace{-3mm}
\end{itemize} 
\\
\hline
Usability Tests & 
These are done in order to check if the target audience can/does use the product in question as intended.\newline
\textcolor{orange}{Often designers have a use case in mind, but users might not directly understand how it is meant to be used, and therefore misuses it.}\newline
In other words, see if the old boomer dude can use the app as well if they are part of the indented audience, or if only a bunch of nerds can use it -> TUI.\\
\hline
\end{tabular}
\end{table}
\pagebreak

\lstdefinelanguage{CSS}{
    sensitive=true,
    keywords={%
    % JavaScript
    typeof, new, true, false, catch, function, return, null, catch, switch, var, if, in, while, do, else, case, break,
    % HTML
    html, title, meta, style, head, body, script, canvas,
    % CSS
    color:, border-radius:, border:, transform:, -moz-transform:, transition-duration:, transition-property:,
    transition-timing-function, background:, background-size:, background-color:, background-image:, background-origin:, background-repeat:, background-position:, background-attachement:, border:, border-box:, border-width:, border-color:, border-bottom:, border-style:, border-radius:, border-spacing:, border-collapse:, text-transform:, text-decoration-thickness:, text-align:, text-indent:, text-shadow:, text-justify:, text-overflow:, text-decoration:, text-align-last:, text-decoration-line:, text-decoration-color:, text-decoration-style:, margin:, padding:, 
    },
    % http://texblog.org/tag/otherkeywords/
    otherkeywords={<, >},   
    ndkeywords={class, export, boolean, throw, implements, import, this},   
    comment=[s]{/*}{*/},
    morecomment=[l]//,
    morecomment=[s]{<!}{>},
    morestring=[b]',
    morestring=[b]",    
    alsoletter={-},
    alsodigit={:}
}
\lstset{
    language=CSS,
    style=code,
}
%%%%%

\begin{table}[h!]
\section{CSS}
\subsection{Syntax}
\begin{tabular}{|m{0.35\linewidth}|m{0.605\linewidth}|}

\hline
%\begin{csscode}
\begin{lstlisting}
body { 
    color: red;
}
\end{lstlisting}
%\end{csscode}
&
The first part is the element you want to modify, this can be the body, header, footer, a class like .button or even everything like with *. You can also add them together like: body, header, .button. \newline
\textbf{All selectors: \#id .class element.class * element element[ attribute ]}\\

\hline
\begin{lstlisting}
/*inline css*/
<p style="color:red;">
    This is a paragraph.
</p>
\end{lstlisting}
&
You can also use css right inside html if you for some reason want to do this. But please only do this when you have a single thing that needs something simple like a color. \newline For everything else use stylesheets.
\\

\hline
\begin{lstlisting}
/*full css inside html*/
<style>
body {
  background-color: linen;
}
</style>
\end{lstlisting}
&
You can of course also use a stylesheet inside html, also pretty useless, just use a proper stylesheet.
\\

\hline
\begin{lstlisting}
/*This is a comment*/ 
/*This is a 
multiline comment*/
// This is also a comment
\end{lstlisting}
&
The comments are made with these /* */ or //. Multilines are also allowed.
\\
\hline
\begin{lstlisting}
color: value|initial|inherit;
\end{lstlisting}
&
Besides the actual values, you can always use initial for the default values, or inherit for the value of the parent.\\
\hline
\begin{lstlisting}
p {
color: red;
}

div {
color: blue;
}

div p {
color: green;
}
\end{lstlisting} &
\textcolor{red}{\textbf{\emph{specifity:}}} \newline
If you have 2 rules that affect 1 element. For example color red and blue, then the most specific rule will win.\newline
For example, consider a paragraph inside a div.\newline
In this case div p is the most specific rule, then followed by p and div.\newline
Aka order: \textcolor{green}{green} \textcolor{red}{red} \textcolor{blue}{blue}\\
\hline
\begin{lstlisting}
h1 {
font-weight: bold;
font-size: 20px; !important;
}
h1 {
font-size: 30px;
}
\end{lstlisting}
& You can override rules with \textbf{\textcolor{red}{!important}} if you specifically want a lower case rule to take effect instead!\newline
\textcolor{red}{Should always be avoided if you can!}\\
\hline
\begin{lstlisting}
<link rel="stylesheet" href="style.css">
<link rel="stylesheet" href="https://shitgaem.online">
\end{lstlisting} & 
Proper link styles can be inserted with <link>. You can also embed a stylesheet from the internet!\\
\hline
\begin{lstlisting}
tr.warning td:nth-child(1)::before{
  color: red;
  content: "\26A0";
}
\end{lstlisting}
& This selects the td of a tr in class warning. Then goes to the nth-child, here 1 and before the content of said td.\newline
Then you can add new content with content: "something";\newline
Keep in mind that any changes with color etc are only made to this new content,\newline 
as you specify it above!
\\
\hline
\begin{lstlisting}
some-element::before {
  display: none;Glace
}
\end{lstlisting}
& If you would like to not display something from the html files, you can do this by specifying
display none before the content in question. \newline Keep in mind that this does not remove the content altogether, it only hides it!
\\
\hline
Inheritance & 
Inheritance in css is only applied sparsely, the reason for this is that you can still overwrite it with pretty much any selector, no matter how low the specificity is.\newline
\textcolor{orange}{If you want something to use inheritance then you need to explicitly state this!}\\
\hline
\end{tabular}
\begin{tabular}{|m{0.977\linewidth}|}
\hline
\textbf{Important note for CSS. If you want to affect an elements position, you will always have to change the attributes in the parent. \newline
In other words, justify-content and similar need to be defined in the box parent, not the div that you want to center!}\\
\hline
\end{tabular}
\end{table}
\pagebreak
\begin{table}[h!]
\subsection{Colors}
\begin{tabular}{|m{0.205\linewidth}|m{0.75\linewidth}|}

\hline
\begin{lstlisting}
color: red;
/* or */
color: #55667788;
/* #RRGGBBAA */
\end{lstlisting}
&
Colors are straight forward, you can either write the name of preconfigured colors, or use the RBGA values. \newline A is opacity.
\\

\hline
\end{tabular}
\subsection{Background}
\begin{tabular}{|m{0.975\linewidth}|}
\hline
\begin{lstlisting}
background-color: red;
background-image: "/path"; /* can also be an url -> url(https://...)*/
background-size: cover|auto|length|percentage|contain;
// cover: resizes to parent, auto: size of image, contain: resize image to fit inside parent, initial: fixed x,y value
background-position: x-value y-value; // top right, top center, center center, ...   Can also be x% y% default 0% 0%.
background-attachement: scroll|fixed|local; // local: scrolls with element, scroll is default.
// this decides if the background scrolls with the page or not.
background-origin: padding-box|content-box|border-box; 
// Used to define where the background image starts inside a box, only works when background-attachement is not fixed.
background-repeat: repeat|repeat-x|repeat-y|no-repeat|space|round; //useless shit
\end{lstlisting}
\\

\hline
\end{tabular}
\subsection{Borders}
\begin{tabular}{|m{0.975\linewidth}|}
\hline
\begin{lstlisting}
border: 4px solid red;
border: border-width border-style border-color;
border-style: none|hidden|dotted|dashed|solid|double|groove|ridge|inset|outset|initial|inherit;
//styles like ridge, groove, inset and outset use 3D effects, hard to describe.
border-radius: 4px 4px 4px 4px; // border-radius: 5px; also possible
//borders can be addressed individually by specifying a direction:
border-bottom: 4px solid red;
border-collapse: separate|collapse|initial|inherit;
// collapse means borders will merge in boxes. seperate will draw borders for both elements.
border-spacing: 4px; //Spacing between borders if the border-collapse is enabled!
\end{lstlisting}\\
\hline
\end{tabular}
\subsection{Margins and Paddings}
\begin{tabular}{|m{0.355\linewidth}|m{0.6\linewidth}|}
\hline
\begin{lstlisting}
margin: top right bottom left;
margin: 10px 10px 10px 10px;
padding: top right bottom left;
padding: top right bottom left;
//margin-right padding-left etc work as well.
\end{lstlisting}
&
Margin is the spacing between the element itself and it's parent and or other elements in the same hierarchy.\newline
Padding is the spacing inside the element to the contents of said element.
\\

\hline
\end{tabular}
\end{table}
\pagebreak
\begin{table}[h!]
\subsection{Text}
\begin{tabular}{|m{0.975\linewidth}|}
\hline
\begin{lstlisting}
text-align: left|right|center|justify; // justify is text as a block.
text-align-last: left|right|center|justify; //The last line can be configured indiidually.
text-justify: auto|inter-word|inter-character|none; /* only used when text-align is set to justify.
inter-word increases spacing between words, inter-character for characters.*/
text-decoration: text-decoration-line text-decoration-color text-decoration-style text-decoration-thickness;
text-decoration-line: none|underline|overline|line-through;
text-decoration-style: solid|double|dotted|dashed|wavy;
text-decoration-thickness: auto|from-font|value;
text-indent: value;
text-overflow: clip|ellipsis|string; 
//clip will just cut the string off, ellipsis ends with ..., a string will display itself.
text-shadow: h-shadow v-shadow blur-radius color|none;
text-transform: none|capitalize|uppercase|lowercase;
\end{lstlisting}
\\
\hline
\end{tabular}
\end{table}
\pagebreak
\begin{table}[h!]
\subsubsection{Fonts}
\begin{tabular}{|m{0.975\linewidth}|}
\hline
\begin{lstlisting}
font: font-style font-variant font-weight font-size/line-height font-family
|caption|icon|menu|message-box|small-caption|status-bar; //use the font used by this element
font-style: normal|italic|oblique; //oblique seems to be the same as italic.
font-variant: normal|small-caps;
font-weight: normal|bold|bolder|lighter|number; //number in 100 steps -> 100,200,300,...,900
font-size: medium|xx-small|x-small|small|large|x-large|xx-large|smaller|larger|length;
line-height: normal|number|length;
font-family: family-name|generic-family; //generic-family example: Isoveka
\end{lstlisting}
\\

\hline
\end{tabular}
\subsection{Display, justify and align}
\begin{tabular}{|m{0.975\linewidth}|}
\hline
\begin{lstlisting}
display: none|inline|block|contents|flex|grid|inline-block|inline-flex|inline-grid|inline-table|list-item|run-in|table|
         table-caption|table-column-group|table-footer-group|table-row-group|table-cell|table-column|table-row;
\end{lstlisting}\\
\hline
Center Div or other HORIZONTALLY \newline 
You can center a box horizontally by using the \textbf{display: flex or display: grid} on the parent box, as well as \textbf{justify-content: center} on the parent box\\
\hline
Center Div or other VERTICALLY \newline 
You can center a box vertically by using the \textbf{display: table} on the \emph{parent of the parent} and using \textbf{display: table-cell} on the parent. \\
\hline
\end{tabular}
\subsection{Selectors and Combinators}
\begin{tabular}{|m{0.3\linewidth}|m{0.655\linewidth}|}
\hline
\textbf{Selectors}\newline
\begin{lstlisting}
selector::pseudo-element {
property: value;
}
\end{lstlisting}
& Pseudo selectors can be used to select only a specific part of an element.\newline
\pic{2022-10-04-10:58:46.png}\\
\hline
\textbf{Combinators and selectors} & 
\minipg{\pic{2022-10-04-11:03:12.png}}{\pic{2022-10-04-11:03:26.png}}[0.35,0.5]\\
\hline
\begin{lstlisting}
:is (box, button, something) p:hover {
background-color: red;
something-else: value;
}
\end{lstlisting}
& This applies a rule to all button box and something elements if hovered.\\
\hline
Selectors in JS & 
\textbf{You can't use the CSS selectors in JS without some weird hacks, therefore, JS has some specific elements inbuilt like first-child, last-child and more.}\\
\hline
\begin{lstlisting}

\end{lstlisting}\\
\hline
\end{tabular}
\section{boxes}
\begin{tabular}{|m{0.3\linewidth}|m{0.655\linewidth}|}
\hline
\textbf{\emph{content-box and border box}} & \minipg{
  \textbf{content-box} increases the box size to ensure the child \textbf{element has the set size}.\newline
  \textbf{border-box} decreases the element size to ensure \textbf{the box itself has the set size}.
}{\includegraphics[scale=0.3]{2022-10-04-11:35:35.png}}[0.3,0.5]\\
\hline
\end{tabular}
\end{table}
\begin{table}[h!]
\begin{tabular}{|m{0,205\linewidth}|m{0.75\linewidth}|}
\hline
\mc{}\\
\hline
\end{tabular}
\end{table}
\begin{table}[h!]
\begin{tabular}{|m{0,205\linewidth}|m{0.75\linewidth}|}
\hline
\mc{}\\
\hline
\end{tabular}
\end{table}
\begin{table}[h!]
\begin{tabular}{|m{0,205\linewidth}|m{0.75\linewidth}|}
\hline
\mc{}\\
\hline
\mc{display: flex} & \mc{dynamically allocates spaces for elements inside this element.
usually combined with justify-content to adjust element position.} \\
\hline
\mc{min-height and min-width} & \mc{This should be used instead of width and height because an object might grow bigger than the max size. This will eventually break the look of the page, or break the application -> see eww.} \\
\hline
\end{tabular}
\end{table}
\pagebreak

\lstdefinelanguage{HTML}{
    sensitive=true,
    keywords={%
    % JavaScript
    typeof, new, true, false, catch, function, return, null, catch, switch, var, if, in, while, do, else, case, break,
    % HTML
    html, title, meta, style, head, body, script, canvas,
    % CSS
    color:, border-radius:, border:, transform:, -moz-transform:, transition-duration:, transition-property:,
    transition-timing-function, background:, background-size:, background-color:, background-image:, background-origin:, background-repeat:, background-position:, background-attachement:, border:, border-box:, border-width:, border-color:, border-bottom:, border-style:, border-radius:, border-spacing:, border-collapse:, text-transform:, text-decoration-thickness:, text-align:, text-indent:, text-shadow:, text-justify:, text-overflow:, text-decoration:, text-align-last:, text-decoration-line:, text-decoration-color:, text-decoration-style:, margin:, padding:, 
    },
    % http://texblog.org/tag/otherkeywords/
    ndkeywords={class, export, boolean, throw, implements, import, this},   
    comment=[s]{/*}{*/},
    morecomment=[l]//,
    morecomment=[s]{<!--}{-->},
    morestring=[b]',
    morestring=[b]",    
    alsoletter={-},
    %otherkeywords={<, >},   
    alsodigit={:}
}
\lstset{
    language=HTML,
    style=code,
}
%%%%%

\begin{table}[h!]
\section{HTML Hypertext Markup Lanugage}
\begin{tabular}{|m{0.975\linewidth}|}
\hline
\textbf{\emph{HTML is not a programming lanugage}}, it is a markup language. It doesn't have logical operations, just as css does't have it. If you want more than a few pictures and a bit of text, then you need javascript or some other actual language to do the work!\newline
HTML simply defines \textbf{content} and \textbf{structure}.\\
\hline
\end{tabular}
\section{Basic Syntax and Information}
\begin{tabular}{|m{0.4\linewidth}|m{0.555\linewidth}|}
\hline
\begin{lstlisting}
<!DOCTYPE html>
<html lang="en">
<head>
    <meta charset="UTF-8">
    <title>hello</title>
</head>
<body>
    Hello World
</body>
</html>
\end{lstlisting}
&
A Simple html page doesn't need even a header or a body, however it is typical of a page to have it.\newline
In the header you can specify title page, encoding languages and more.\newline
Content is typically shown inside the body, while addresses and Information is usually found in the footer.\newline
the <> </> open and close a tag, usually tags need both, but some may only need an opening tag.
\\
\hline
\begin{lstlisting}
<!--This is an html comment!-->
<!--This is a 
multiline comment!-->
\end{lstlisting}
&
Comments are made with <!-- -->.
\\
\hline
\begin{lstlisting}
<input type="text" placeholder"enter something">
\end{lstlisting}
&
This is an empty tag, no closing </> is allowed. \newline
The optional / at the end is \textbf{strongly discouraged} -> <input type="text"/>
\\
\hline
\begin{lstlisting}
<div>Some text!<div>
\end{lstlisting}
&
Tags are opened with <> and closed with </>.
\\
\hline
\vspace{1.5mm}
\pic{2022-09-27-10:53:50.png}
&
Certain elements like <title> are only allowed in the header, others like <h> are only allowed in the body.\newline
You can technically just use these tags without specifying the body and header.\newline
However this might be confusing to some people.
\\
\hline
Semantic Markup
&
-- In order to provide better support for people with accessibility issues, use proper html,\newline \,\,\, this means use proper lists and don't use divs that you force-convert to lists.\newline
-- It also helps with recognition in search engines.\newline
-- Mobile support is also likely to be better with proper html.
\\
\hline
\end{tabular}
\section{Base Elements}
\begin{tabular}{|m{0.755\linewidth}|m{0.2\linewidth}|}
\hline
\begin{lstlisting}
<img src="path to source" alt="alternative text">
\end{lstlisting}
& Inserts an image in html.\\
\hline
\begin{lstlisting}
<input type="text" placeholder="something">
\end{lstlisting}
& Input with placeholder\\
\hline
\begin{lstlisting}
<form onsubmit={() => handleSubmit() }><!--Inputs and buttons.--> </form>
\end{lstlisting}
& Form with js function\\
\hline
\begin{lstlisting}
<h1>Some text to use as header</h1> <!--h1,h2,h3,h4,h5,h6-->
\end{lstlisting}
& Header-text, from 1 to 6.\\
\hline
\begin{lstlisting}
<p>Some text to display</p>
\end{lstlisting}
& Paragraph\\
\hline
\begin{lstlisting}
<ol>
    <li>element1</li>
    <li>element2</li>
    <li>element3</li>
</ol>
\end{lstlisting}
& Ordered List.\\
\hline
\begin{lstlisting}
<ul>
    <li>element1</li>
    <li>element2</li>
    <li>element3</li>
</ul>
\end{lstlisting}
& Unordered List.\\
\hline
\begin{lstlisting}
<a href="https://shitgaem.online">Shitgaem</a>
\end{lstlisting}
& Hyperlink\\
\hline
\begin{lstlisting}
<code>std::cout << "Henlo Birb!\n"</code>
\end{lstlisting}
& Did someone say minted?\\
\hline
\begin{lstlisting}
<b>this text is bold, or maybe not.</b>
\end{lstlisting}
& \textbf{bold text}\\
\hline
\begin{lstlisting}
<em>This text is italic, or maybe not</em>
\end{lstlisting}
& \emph{italic text}\\
\hline
\begin{lstlisting}
<div>This is the standard div container, try centering it!</div>
\end{lstlisting}
&
div
\\
\hline
\begin{lstlisting}
 <nav>
  <a href="/html/">HTML</a> |
  <a href="/css/">CSS</a> |
  <a href="/js/">JavaScript</a> |
  <a href="/cpp/">C++</a>
</nav>
\end{lstlisting}
&
Navigation Bar
\\
\hline
\end{tabular}
\end{table}
\pagebreak
\begin{table}[h!]
\begin{tabular}{|m{0.755\linewidth}|m{0.2\linewidth}|}
\hline
\begin{lstlisting}
<aside>Something to display on the side</aside>
\end{lstlisting}
& Implements a bar on the side.\\

\hline
\end{tabular}
\subsection{Semantic Element, what to choose?}
\includegraphics[scale=0.4]{2022-09-27-11:45:58.png}
\section{Global Attritubes}
\begin{tabular}{|p{0,2\linewidth}|p{0.755\linewidth}|}
\hline
accesskey & Specifies a shortcut key to activate/focus an element\\
\hline
class & Specifies one or more classnames for an element (refers to a class in a style sheet)\\
\hline
contenteditable & Specifies whether the content of an element is editable or not\\
\hline
data-* & Used to store custom data private to the page or application\newline
\textcolor{orange}{These (for example) used to handle event bubbling\newline 
data-event= add -> handle add in parent node to rerender, etc\newline
In the parent node you can then perform a check on this by using target.class == add\newline
\textbf{This makes sure you don't use the same function for every single bubbling event!}}\newline
\begin{lstlisting}
document.querySelector("pingpang").addEventListener("click", (event) => { 
  if (event.target.getAttribute("data-eventtype") == "add") {
   rerender();
  } else {
    console.log("pingpang!");
  }
})
\end{lstlisting}
\\
\hline
dir & Specifies the text direction for the content in an element\\
\hline
draggable & Specifies whether an element is draggable or not\\
\hline
hidden &	Specifies that an element is not yet, or is no longer, relevant\\
\hline
id & Specifies a unique id for an element\\
\hline
lang & Specifies the language of the element's content\\
\hline
spellcheck & Specifies whether the element is to have its spelling and grammar checked or not\\
\hline
style & Specifies an inline CSS style for an element\\
\hline
tabindex & Specifies the tabbing order of an element\\
\hline
title & Specifies extra information about an element\\
\hline
translate & Specifies whether the content of an element should be translated or not\\
\hline
\textbf{\emph{Tag Omission}} & You can omit certain tags. Namely the following: <html>, <body>, <head>, <p>, <li>.\newline
It is generally recommended to do this for stylistic reasons.\\
\hline
\end{tabular}
\section{Script in HTML}
\begin{tabular}{|m{0.2\linewidth}|m{0.755\linewidth}|}
\hline
<script> & 
\textcolor{orange}{To use js inside html you need to be sure that the DOM is already loaded}\newline
\begin{lstlisting}
<head>
<!-- the button can't be accessed here! -->
</head>
<body>
<button>clickme</button>
<script> 
// some js script for the button
// make sure to place it here
</script>
</body>
\end{lstlisting}\\
\hline
\end{tabular}
\end{table}
\pagebreak
\begin{table}[ht!]
\section{Web Server}
\begin{tabular}{|m{0.2\linewidth}|m{0.755\linewidth}|}
\hline
Https Request & 
Https uses requests to both get and push data.\newline
We call them \textbf{GET request} and \textbf{POST request}.\newline
Usually we only use GET requests in combination with js like this:\newline
\begin{lstlisting}
async function fetchFromURL() {
  try {
    const response = await fetch('result.json', { method: 'GET'});
    if(response.ok) {
      const content = await response.text();
      console.log(content);
    }
  } catch(e) {
    // erroreroni
  }
}
\end{lstlisting}\\
\hline
\end{tabular}
\end{table}
\end{document}

